\begin{opinion}
    El trabajo Módulo fabric-ca-client del SDK de Fabric para gestionar identidades criptogr\'aficas con C\# desarrollado por la estudiante Gabriela B. Martínez Giraldo, cumple con los requisitos para la culminación de la carrera de Ciencia de la Computación de la Universidad de La Habana.
   
    Este módulo es un componente de un SDK implementado en el lenguaje de programación C\# para interactuar con la tecnología de blockchain Hyperledger Fabric (HF). Siendo Hyperledger Fabric una de las plataformas usadas para el desarrollo de nuestras aplicaciones blockchain y siendo una debilidad la ausencia de bibliotecas, SDK y API en C\# hacen pertinente y útil el desarrollo de este módulo.
    
    Es válido aclarar que la diplomante ha mostrado excelentes habilidades técnicas durante el desarrollo del trabajo, demostrando interés, dominio del tema y cumpliendo con todos los requisitos definidos por el cliente. Para ello, comenzó con la asimilación y estudio de las tecnologías indicadas por los tutores, mostrando además buenas capacidades de asimilación e independencia
    
    Por tanto, felicitamos a la estudiante por la labor desarrollada y consideramos que la tesis reúne los estándares metodológicos exigidos por la Facultad de Matemática y Computación de la Universidad de la Habana, para ser presentada y sometida a evaluación en su ejercicio de defensa.
    
    Exhortamos a la estudiante a trabajar para convertir este trabajo en una publicación.
    
    La Habana, enero del 2022\\
    
    Camilo Denis González $\_\_\_\_\_\_\_\_\_\_\_\_\_\_\_$\\
    
    Camilo Denis González $\_\_\_\_\_\_\_\_\_\_\_\_\_\_\_$\\
    
    Miguel Katrib Mora $\_\_\_\_\_\_\_\_\_\_\_\_\_\_\_\_\_$\\
\end{opinion}