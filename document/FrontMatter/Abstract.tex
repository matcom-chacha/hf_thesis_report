\begin{resumen}
\emph{Hyperledger Fabric (HLF)} constituye una tecnolog\'ia de redes distribuidas (DLT) dise\~nada espec\'ificamente para el sector empresarial. 
%Resulta una de las mejores plataformas disponibles en t\'erminos de procesamiento de transacciones y latencia de confirmaci\'on de las mismas; y es ampliamente reconocida por su arquitectura modular y altamente configurable.

Una de sus peculiaridades m\'as relevantes, radica en el hecho de ser la primera DLT en brindar soporte para programar contratos inteligentes en lenguajes de prop\'osito general, en lugar de limitar el desarrollo a lenguajes espec\'ificos al dominio (DSL). Actualmente, ofrece la capacidad de implementar smart contracts en Go, Java y Node.js.

Por las amplias ventajas que esta plataforma reporta, resulta de inter\'es extender el conjunto de lenguajes a los que ofrece soporte.

El presente trabajo propone el diseño e implementaci\'on de un m\'odulo \emph{fabric-ca-client} de un \emph{Fabric-SDK} en C\#, con el fin de poder interactuar con la Fabric CA de HLF utilizando este lenguaje. 
%Este proyecto forma parte de uno m\'as ambicioso que aspira a poder desarrollar chaincode en C\#.
\end{resumen}

\begin{abstract}
	\emph{Hyperledger Fabric} (HLF) is a distributed networking technology (DLT)
	designed specifically for the business sector.
	
	One of its most relevant peculiarities lies in the fact that it is the first DLT
	in providing support for programming smart contracts in general purpose languages, rather than limiting development to domain-specific ones(DSL).
	
	It currently offers the ability to implement smart contracts in Go, Java, and Node.js.
	
	Due to the wide advantages this platform reports, it results of interest to extend the set of languages it supports.
	
	This paper proposes the design and implementation ot a \texttt{fabric-ca-client} module of a \emph{Fabric-SDK} written in C\#, in order to interact with a HLF's CA usign this language.
\end{abstract}

