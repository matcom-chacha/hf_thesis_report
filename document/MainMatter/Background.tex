\chapter{Antecedentes}\label{chapter:state-of-the-art}

En el presente cap\'itulo se ofrecen una serie de conceptos relevantes al tema en cuesti\'on. Su prop\'osito es el de facilitar la comprensi\'on de las bases en las que se enmarca esta investigaci\'on. %este proyecto

La mayor\'ia de las definiciones que se proveen tienen como fuente al \emph{paper} de Hyperledger Fabric [\cite{androulaki2018hyperledger}], la documentaci\'on oficial de esta blockchain [\cite{hyperledgerorg}] y la de Fabric CA [\cite{fabricca}]. Para el resto se esclarecen las referencias apropiadas.%conceptos, se presentan, definiciones provienen de % adecuadas % debidamente referenciados %referencias correspondientes

\section{\textquestiondown Qu\'e se entiende por Blockchain?}

Una blockchain o cadena de bloques no es m\'as que un libro mayor de transacciones inmutable y distribuido entre una red de nodos. Cada nodo mantiene una copia del \emph{ledger}, la cual se actualiza por medio de transacciones ejecutadas y validadas a partir de un protocolo de consenso. La informaci\'on (transacciones) se agrupa en bloques que incluyen un hash del bloque precedente, de forma que resulta casi imposible alterar los datos de una transacci\'on sin falsificar tambi\'en la de los elementos siguientes.

Entre las ventajas de esta tecnolog\'ia es posible destacar:

\begin{enumerate}
	\item  No censura: Una de las caracter\'isticas m\'as relevantes de las plataformas blockchains es que en ellas no existe una autoridad centralizada con total control sobre la red. Esto es posible gracias a los mecanismos de consenso como \emph{Proof or Work} que aseguran se llegue a un acuerdo respecto al estado del ledger entre todos los nodos.
	
	De esta forma, se garantiza una especie de confianza entre elementos desconocidos de la red y se establece un protocolo que impide que una \'unica parte, ni gubernamental ni de ning\'un otro origen, sea capaz de censurar al resto.
	
	%que e nodo q se annade sea la unica version de la readlidad a partir de lo acordado por todos los nodos en la blockchain
		
	\item Seguridad: A diferencia de lo que es costumbre en las bases de datos tradicionales con las operaciones CRUD (funciones para crear, leer, actualizar y borrar), la informaci\'on que se guarda en una blockchain es persistente e inmutable. Gracias al mecanismo de enlazar bloques a partir de su hash se hace muy complejo que un atacante manipule los datos, puesto que se ver\'ian alterados tambi\'en todos los bloques consecutivos. Dado que la informaci\'on no se respalda en un \'unico destino, sino que se distribuye por todos los nodos de la red, las acciones fraudulentas de este tipo se encuentran un escenario a\'un m\'as complejo de penetrar.
	
	% se previene accesos no autorizaofs a aprtir de la anonimizacion de datos personales y la aplicacion de redes permisionadas
	\item Transparencia: El hecho de que las blockchains sean descentralizadas implica que cualquier miembro pueda visualizar en todo momento los datos que en ellas se guardan. Esto permite comprobar su correctitud, disminuyendo las posibilidades de fraudes que suelen acaecer en otros escenarios.
	
	\item Trazabilidad: Este punto est\'a muy relacionado al anterior. Precisamente por la transparencia de la red es que cada usuario es capaz de verificar el historial de todas las transacciones ejecutadas sobre esta. As\'i, es posible conocer cada estado por el que ha transitado un activo y realizar auditor\'ias para comprobar que todo marche acorde a lo deseado.	

\end{enumerate}

%desventajas
%INFO DEL TIMELINE 
%immutability, privacy, security, and transparency

\section{Nociones B\'asicas de Hyperledger Fabric}

Hyperledger Fabric (HLF) es un proyecto de c\'odigo libre del consorcio Hyperledger, creado su vez por la \emph{Linux Foundation} [\cite{linuxfoundation}] con el objetivo avanzar en el desarrollo de tecnolog\'ias blockchain. Consiste en una DLT dise\~nada espec\'ificamente para entornos empresariales, atendiendo a las necesidades de identificar a los participantes en la red, establecer una serie de permisos para la interacci\'on, de contar con privacidad y confidencialidad de trasacciones, baja latencia en la confirmaci\'on de estas y capacidad para enviar una cantidad elevada de datos por unidad de tiempo (\emph{throughput}).

HLF surge entonces como plataforma privada y permisionada. A diferencia de los sistemas abiertos en los que cualquier desconocido puede consultar y modificar los datos, Fabric utiliza un Proveedor de Servicios de Membres\'ia (MSP) para poder identificar a cada usuario que se una a la red. As\'i, limita la interacci\'on para s\'olo un n\'umero de elementos autorizados y establece facilidades para otorgar permisos espec\'ificos a cada uno. Ofrece adem\'as una caracter\'istica muy novedosa y \'util, d\'igase la utilizaci\'on de canales como especie de subredes para compartir informaci\'on confidencial entre pares o grupos de miembros que lo requieran.

Fabric posee una arquitectura altamente modular y configurable, capaz de ser adaptada a los diversos requerimientos que puedan surgir en la industria. Facilita customizar por ejemplo el protocolo de consenso a utilizar para determinar el orden de las transacciones. Con este fin provee implementaciones de mecanismos crash fault-tolerant (CFT) como Raft (el actual y recomendado), Kafka y Solo (para versiones de prueba), lo cual permite a los usuarios elegir el m\'as adecuado al nivel de confianza existente entre ellos.
%Aqui pueden referenciarse varios articulos sobre los mecanismo de ordenaicon, solo que considero no son tan relevantes al tema tratado

De igual forma cuenta con protocolos enchufables de gesti\'on de identidad como LDAP u OpenID Connect, con protocolos de gesti\'on de claves o bibliotecas criptogr\'aficas, entre otros.

N\'otese adem\'as que, al utilizar mecanismos de consenso deterministas no requiere del uso de criptomonedas para incentivar el minado. Lo anterior resulta en una seguridad incrementada al no atraer atacantes que quieran socavar la estabilidad de la red. Adem\'as implica un costo operacional menos elevado de la plataforma .

\subsection{Componentes principales}

\subsubsection{Ledger}
El ledger o libro mayor de cuentas de Fabric est\'a conformado por dos elementos: Un \emph{world state} o estado mundial y un \emph{transaction log} o lista de trasacciones. El primero resulta ser como la base de datos del ledger, describe el estado de este en un instante de tiempo dado. Por su parte, el segundo, mantiene un registro del historial de trasacciones que han resultado en el estado actual del world state.

Cada participante conserva una copia del ledger por cada red de HLF  a la que pertenece.

\subsubsection{Assets}
Ha de resaltarse que, mientras en blockchains como Bitcoin se mantiene un registro de las transferencias de criptomonedas, en Fabric la definici\'on de \textquestiondown qu\'e es transferido? resulta un poco m\'as flexible. En el \'ultimo caso se definen activos o \emph{assets} a transferir, lo que da cabida a un n\'umero de usos m\'as extendido.
%aqui se puede argumentar

Los activos pueden variar desde elementos tangibles como bienes ra\'ices y hardware, hasta intangibles como contratos y propiedades intelectuales. Hyperledger Fabric brinda la capacidad de modificar activos mediante transacciones de c\'odigo de cadena.

Los activos representan precisamente la colecci\'on de pares clave-valor, cuyos cambios de estado se registran como transacciones en el libro mayor.

\subsubsection{Chaincode}

Los \emph{smart contracts} son llamados \emph{chaincode} o c\'odigo de cadena en Fabric. Estos constituyen la l\'ogica de negocio de una aplicaci\'on blockchain. 

Los chaincodes definen activos y transacciones para modificarlos. Pueden ser desplegados din\'amicamente y correr de forma concurrente en la red.

Los elementos participantes de cada red permisionada en HLF son capaces de utilizarlos para interactuar con el ledger y transformarlo.

\subsubsection{Nodos}

Los nodos constituyen las entidades de comunicaci\'on de una blockchain. En HLF existen 3 tipos:

\begin{enumerate}
	\item Nodos Clientes: Los clientes que env\'ian invocaciones de trasacciones a los \emph{endorsers} y transmiten las proposiciones de trasacciones al servicio de ordenaci\'on.
	
	
	\item Nodos Pares: O \emph{peers}, qui\'enes ejecutan trasacciones y mantienen el estado y una copia del ledger. Pueden asumir el rol de endosadores o endorsers.
	
	\item Nodos Ordenadores: El conjunto de estos nodos conforman lo que se le llama servicio de ordenamiento, el cual se encarga de ordernar las trasacciones endosadas y empaquetarlas en bloques. De igual forma se ocupan de darle un orden a dichos bloques siguiendo un protocolo de consenso como se comentaba anteriormente. Su segundo objetivo consiste en distribuir dichos bloques entre los peers para su posterior validaci\'on.
	
\end{enumerate}

Aqu\'i es importante destacar c\'omo el hecho de separar el endosamiento de ejecuci\'on de chaincode (en los peers) del servicio de ordenamiento elimina el embotellamiento, lo cual le provee a HLF ventajas de desempe\~no y escalabilidad sobre los sistemas que ejecutan ambas fuciones en los mismos nodos.

%arquitectura de HLF?
 
\subsection{Flujo}

La mayor\'ia de las blockchains utilizan un paradigma nombrado \emph{order-execute} que requiere, para alcanzar el consenso, que los smart contracts a utilizar sean deterministas. Lo anterior limita su programaci\'on a lenguajes de dominio espec\'ifico que eliminen las operaciones no deterministas. Adem\'as, require que todas las transacciones se ejecuten secuencialmente por todos los nodos, lo que implica que deban ser tomadas medias de protecci\'on extra ante posibles ataques maliciosos que deseen desequilibrar la red.

En cambio, en Fabric se utiliza un arquitectura llamada \emph{execute-order-validate} que resuelve muchos de los problemas comentados.

Dicho modelo consta de tres pasos:
\begin{enumerate}
	\item Primeramente se ejecuta una trasacci\'on, se chequea su correctitud y se respalda (proceso de \emph{endorsement}).
	
	\item Luego se ordenan las trasacciones a partir del protocolo de consenso.
	
	\item Se validan teniendo en cuenta la pol\'itica de endosamiento espec\'ifica a la aplicaci\'on manejada y, por \'ultimo, se aplican al ledger.
\end{enumerate}

Las pol\'iticas de endosamiento a las que se hace referencia dictan qu\'e nodos par y cu\'antos de ellos deben garantizar la correcta ejecuci\'on de un contrato inteligente determinado. De esta forma se asegura que solo un subcojunto de los peers deba ejecutarlo. Esto facilita que el sistema sea m\'as escalable y tenga un mejor desempe\~no con la capacidad a\~nadida de ejecuciones en paralelo.

Igualmente, ha de notarse c\'omo se elimimina el no determinismo, al poder filtrar los resultados inconsistentes antes de pasar a la fase de ordering.

Dicho cambio de arquitectura contituye una de las caracter\'isticas que realzan la importancia de HLF. Este nuevo modelo que implementa hace posible la programaci\'on de chaincode en lenguajes est\'andar como Python, Java y Javascript. V\'ease su grandeza de facilitar al programador el crear un v\'inculo con esta nueva DLT, ahorr\'andole horas de estudio para aprender un nuevo lenguaje de programaci\'on y de debugueo para encontrar errores nunca tratados por \'el.

Precisamente en esta capacidad de Fabric es en la que se apoya este proyecto, puesto que es la que har\'a posible la programaci\'on de smart contracts en C\# una vez implementado el engranaje apropiado.
 
\section{Mecanismos de autenticaci\'on y\\ permisionamiento en Fabric}
%considrar un nombre mas  apropiado (Servicios de membresia)
Como se ha mencionado anteriormente, una de las caracter\'isticas m\'as destacadas de HLF resulta el hecho de ser una red privada y permisionada. Esto significa que opera sobre un conjunto de participantes verificados que solo pueden interatuar con la red cuando el administrador oportuno le ha concedido los permisos requeridos. 

Bajo este principio de governanza se garantiza cierto nivel de confianza, puesto que, aunque los usuarios no se conozcan entre s\'i, toda acci\'on realizada queda registrada identificando de forma un\'ivoca a su ejecutor. As\'i, en caso de detectarse alguna anomal\'ia, es posible identificar y rastrear al responsable, y tomar las medidas necesarias para solucionar el incidente.

A continuaci\'on se describen con un poco m\'as de profundidad los mecanismos implementados por Fabric para asegurar el correcto funcionamiento de este modelo permisionado.

\subsection{Infraestructura de Clave P\'ublica (PKI)}
Una infraestructura de clave p\'ublica (PKI del ingl\'es Public Key Infraestructure) es una colección de tecnolog\'ias de Internet que proporciona comunicaciones seguras en una red. Enlaza llaves con entidades y provee los servicios necesarios para la correcta administraci\'on de certificados digitales y el cifrado de clave p\'ublica.

Cada PKI cuenta elementos claves como son:
\begin{enumerate}
	\item Autoridad certificadora (CA): Encargada de emitir certificados de clave p\'ublica para una entidad a modo de confirmaci\'on de sus credenciales.
	
	\item Llaves p\'ublicas y privadas: Indispensables para el funcionamiento de los mecanismos de firma digital.
	
	La llave privada que ha de mantenerse secreta, es la utilizada para producir firmas digitales sobre mensajes. La p\'ublica se deja disponible a qui\'en desee consultarla y es la que funciona como ancla de autenticaci\'on. Se establece una relaci\'on matem\'atica entre ambas llaves de forma tal que, la firma producida para un mensaje con una llave privada espec\'ifica solo es v\'alida bajo su correspondiente llave p\'ublica y para el mensaje dado. As\'i, es posible verificar la integridad de la informaci\'on transmitida y la autenticidad de su remitente.
	
	\item Certificados digitales: Documentos que contienen un conjunto de propiedades relacionados con su titular. Incluyen la llave p\'ublica de la parte involucrada y, opcionalmente, atributos del titular de la llave privada (d\'igase roles u otros), fecha de vigencia, y la firma digital de la CA emisora.
	%el tipo de certificado mas comun es el x.509 que permite a la parte que codifica especificar detalles en su estructura
	
	\item Lista de revocaci\'on de certificados: Del ingl\'es Certificate Revocation List (CRL), constituye una lista generada por la CA que referencia a los certificados que ya no son v\'alidos. Pueden ingresar a la lista certificados que hayan sido comprometidos o aquellos cuyo material criptogr\'afico se haya visto expuesto.
	
\end{enumerate}

En Fabric se adopta un modelo jer\'arquico de infraestructura de clave pública tradicional, el cual, junto a una implementaci\'on de MSP permite el manejo correcto de identidades. En los siguientes ep\'igrafes se analiza mejor este engranaje.

\subsection{Autoridades Certificadoras}
Una autoridad certificadora (CA del ingl\'es certificate authority) no es m\'as que una entidad capaz de producir, firmar y almacenar certificados digitales.

Por lo general el flujo que se sigue para el intercambio pasa por:

\begin{enumerate}
	\item Una parte emite una solicitud de certificado digital a partir de la generaci\'on de su llave privada (que no devela ni a la CA), su llave p\'ublica y una solicitud de firma de certificado (CSR del ingl\'es certificate signing request) con la informaci\'on que se requiere plasmar en \'el.
	
	\item La CA chequea que los datos recibidos sean correctos y emite un certificado que firma con su clave privada.
	
	\item Finalmente, terceros pueden verificar la autenticidad de la entidad inicial a partir de la llave p\'ublica de la CA involucrada.
\end{enumerate}


\subsection{Fabric CA}
%Because Fabric CA is a custom CA targeting the Root CA needs of Fabric, it is inherently not capable of providing SSL certificates for general/automatic use in browsers.

Aunque HLF puede utilizar cualquier autoridad certificadora facultada para generar certificados X.509, cuenta con una implementaci\'on propia llamada Fabric CA. Su uso es opcional, no obstante, se recomienda elegir esta alternativa por estar capacitada para la creaci\'on de la estructura de carpetas MSP local y la organizaci\'on requerida por HLF.%certificados ECDSA ademas de haber sido testeada y comprobada su eficacia en este sentido

Provee funcionalidades tales como:
\begin{enumerate}
	\item Registro de entidades o capacidad de conexi\'on a un protocolo ligero de acceso a directorios (LDAP o \emph{Lightweight Directory Access Protocol}) que funcione como registro de usuarios.
	
	\item Emisi\'on de certificados de enrolamiento o incripci\'on para firmas e identificaciones.%ECerts enrollment certf
	
	\item Expedici\'on de certificados de transacci\'on.
	
	\item Anonimato y desvinculaci\'on al realizar transacciones.
	
	\item Renovaci\'on y revocaci\'on de certificados.
\end{enumerate}

La Fabric CA est\'a conformada por dos componentes: un servidor y un cliente. En la imagen \ref{fig:cadiagram} se puede apreciar c\'omo el servidor Fabric CA encaja en la arquitectura general de Hyperledger Fabric.

\begin{figure}[h]
	\centering
	\includegraphics[scale=0.3]{Graphics/fabric-ca}
	\caption{Fabric CA en comunicaci\'on con una red de Hyperledger Fabric.}
	\label{fig:cadiagram}
\end{figure}

Para interactuar con la Fabric CA se pueden utilizar o bien el Fabric CA client [\cite{caclient}] o el SDK correspondiente al lenguaje con el que se desarrolle la aplicaci\'on en concreto. Todas la comunicaciones se realizan a trav\'es de de REST APIs [\cite{restapifaclient}], s\'olo que los componentes anteriores funcionan como interfaces al servidor. Resultan una envoltura m\'as adecuada para los no tan amigables pedidos REST.

El cliente o el fabric sdk pueden conectarse a un servidor en un grupo de servidores de HLF. En la imagen \ref{fig:cadiagram} se puede observar c\'omo el cliente accede a un \emph{endpoint} (HA Proxy) que equilibra la carga del tr\'afico hacia uno de los miembros del cl\'uster. Todos los miembros comparten una misma base de datos para el almacenamiento de identidades y certificados. En caso de configurarse el LDAP, la informaci\'on referente a las identidades se mantiene en un registro de este tipo.

Por defecto, el servidor de Fabric CA cuenta con una solo CA. No obstante, es posible configurarlo para que haga uso de m\'ultiples CAs, encaden\'andolas entre s\'i para establecer una red de mayor confianza. As\'i, los certificados ra\'ices pueden ser expedidos por una CA padre u otra intermediaria, lo que permite que la red escale m\'as f\'acilmente y protege a su CA raiz.%incluso es posible deshabilitar la root ca una vez las intermediarias esten en funcionamiento, lo que reduce su vulnerabilidad

Para cada red de HLF se recomienda el despliegue de dos autoridades certificadoras por organizaci\'on, una CA de organizaci\'on y una CA de TLS (\emph{Transport Layer Security}). Aunque estas no difieren en funcionamiento, s\'i lo hacen el tipo de certificado que expiden. La primera es la encargada de generar identidades para nodos y organizaciones, mientras que la segunda es la que asegura las comunicaciones en la red.

Por lo general se aconseja tambi\'en separar al servicio de ordenamiento en una organizaci\'on diferente a las de los nodos pares con su propia CA. En caso de existir nodos ordenadores agrupados por organizaciones, ha de garantizarse una CA propia para cada una.

Toda esta separaci\'on de funcionalidades permite distribuir las gestiones lo m\'as posible, lo cual reduce las posibilidades que tendr\'ia un atacante ante la red y resulta indispensable en general para minimizar las vulnerabilidades la blockchain.

%garantizando que el material criptogr\'afico que proveen estas autoridades sea de total confianza.

\subsection{MSP}
Mientras las autoridades certificadoras generan identidades a partir de un par de llaves p\'ublicas y privadas, es necesario un mecanismo para reconocer dichas entidades en la red. Aqu\'i es donde aparece en escena el concepto de Provedoores de Servicios de Membres\'ia o MSP del ingl\'es \emph{Membership Service Providers}.

Los MSP son componentes de HLF que definen las reglas que rigen a las entidades v\'alidas para una organizaci\'on espec\'ifica. Permiten convertir identidades en roles; son qui\'enes determinan qu\'e permisos tiene cada entidad a nivel de organizaci\'on, nodo y canal; y definen las organizaciones (inclu\'idas Cas) en las que conf\'ian los miembros de una red. Por ello, se consideran encargados de validar las acciones realizadas por cada entidad, velando siempre porque cada una realice solo aquellas a las que se le ha dado acceso.

A diferencia de lo que su nombre puediera indicar, un MSP no provee nada. Su implementaci\'on concreta consiste en una serie de carpetas que se a\~naden a la configuraci\'on de la red y permite definir a cada organizaci\'on tando a nivel interno (declarando por ejemplo a sus administradores) como externo (permitiendo que otras organizaciones validen que las entidades posean la autoridad para realizar las acciones que solicitan).

Existen dos tipos de configuraciones MSP, una a nivel local y otra de canal. La divisi\'on entre estas refleja las necesidades de las organizaciones para administrar sus recursos locales, como un par o nodos de pedido, y sus recursos de canal, como contratos inteligentes y otros que operen a nivel de canal o red.

Una carpeta MSP local est\'a compuesta por:

\begin{enumerate}
	\item Un archivo \emph{config.yaml} utilizado para configurar la clasificaci\'on de identidades en Fabric al habilitar el \emph{Node OUs} y definir de los roles aceptables.
	
	\item Una carpeta \emph{cacerts} con una lista de certificados autofirmados X.509 de las CAs ra\'iz en las que conf\'ia la organizaci\'on a la que representa el MSP.
	
	\item Una carpeta \emph{intermediatecerts} con certificados X.509 de las CAs intermedias en las que conf\'ia la organizaci\'on. Cada certificado debe estar firmado por un Ca de la carpeta de ra\'ices confiables o bien otro intermediario cuya cadena termine en el mismo sitio. Ambas carpetas definen las CAs de las que deben provenir los certificados para considerarse miembros de la organizaci\'on, aunque a diferencia de la primera, la de intermediarios puede estar vac\'ia.
	
	\item Una carpeta de \emph{admincerts} que lista las identidades que definen actores con el rol de administradores en la organizaci\'on. En versiones posteriores a la 1.4.3 no existe; en cambio, se definen los administradores en el momento de creaci\'on de la entidad, cuando la CA determina el valor de la propiedad \emph{Node OU rol} del certificado.
	
	
	\item Una carpeta \emph{keystore} con la llave privada del nodo correspondiente (ya sea par, ordenador o cliente). Dada su utilidad para la firma de trasacciones y respuestas se limita su acceso a los administradores del nodo en s\'i. 
	
	\item Para nodos pares u ordenadores una \emph{signcert} que contine el certificado generado por la CA. Representa su identidad y junto a su llave privada correspondiente puede ser utilizada para la creaci\'on de firmas.
	
	\item Un par de carpetas \emph{tlscacerts} y \emph{tlsintermediatecerts} \'idem a las 2 y 3 pero con los certificados de las CAs para asegurar las comunicaciones con TLS.
	
	\item Por \'ultimo, una carpeta \emph{operationcerts} con los certificados requeridos para establecer comunicaci\'on con la API de \emph{Fabric Operations Service} (destinada a los operadores de la red).
\end{enumerate}

Puesto que en el caso del MSP de canal no se necesitan habilidades de firma sino de verificaci\'on de entidades se excluyen los directorios de keystore y signcert. Adem\'as se a\~nade una carpeta extra, \emph{Revokes Certificates}, con informaci\'on referente a las entidades de actores que hayan sido revocadas. Para certificados del tipo X.509 se guardan pares de \emph{strings} conocidos como \emph{Subject Key Identifier} (SKI) y \emph{Authority Access Identifier} (AAI). Dicha lista es conceptualmente id\'entica a la CRL de una CA, pero se relaciona tambi\'en con la revocaci\'on de membres\'ia de la organizaci\'on.


\section{Fabric SDKs y Fabric Gateways} 

Las aplicaciones que deseen interactuar con el ledger han de hacerlo por medio de un intermediario, un \emph{Software Development Kit (SDK)} o paquete de desarrollo de \emph{software}. Los espec\'ificos a HLF est\'an dise\~nados para comunicarse con los smart contracts, encargados a su vez de realizar las operaciones de \emph{get, put, delete} sobre el world state y registrar las trasacciones en la blockchain.%deben hacerlo

Hyperledger Fabric ofrece un n\'umero de SDKs para el desarrollo de aplicaciones en una amplia variedad de lenguajes de programaci\'on. Actualmente cuenta con soporte para Node.js [\cite{sdknode}] y Java [\cite{sdkjava}], los primeros en aparecer. Igualmente, a pesar de a\'un no haber sido publicados de forma oficial, expone para su descarga y prueba implementaciones para Go [\cite{sdkgo}] y Python [\cite{sdkpython}].

Vale aclarar que estos paquetes son considerados SDKs de bajo nivel. Contienen una considerable cantidad de c\'odigo y, por tanto, una complejidad bastante elevada. Queda en manos programador con la intenci\'on de conectar una aplicaci\'on con Fabric, comprender el funcionamiento de las solicitudes de transacciones y el endosamiento para implementarlos a partir de estas plataformas, dado que en realidad lo que contienen no es m\'as que una adaptaci\'on de las llamadas gRPC [\cite{grpc}] para cada lenguaje espec\'ifico.

Para resolver estos problemas se han creado APIs de alto nivel que le permiten a los desarrolladores abstraerse un poco m\'as del funcionamiento de HLF en sus or\'igenes. As\'i, con proyectos como [\cite{fabricgatewayjava}] y [\cite{fabricgateway}] se facilita la realizaci\'on de consultas, ejecuci\'on de transacciones y respuesta a eventos que permitan conocer y modificar el estado del ledger.

El surgimiento de estos \'ultimos modelos m\'as consistentes, supresores de mucho c\'odigo repetitivo propenso a errores, ha sido bien recibido por la comunidad. No obstante, su funcionamiento cuenta en las bases con los SDKs mencionados al inicio, siendo de hecho construidos encima de estos. Adem\'as carecen de un mecanismo para gestionar tareas de administraci\'on. Por ello, a\'un se hace necesario intercambiar directamente con los SDKs de bajo nivel para la creaci\'on de identidades, su registro entre otras operaciones de relevancia.

Dado que el objetivo del presente trabajo gira en torno a estas funcionales, son entonces los SDKs originales de Java, Node, Go y Python los que fungen como gu\'ia para el desarrollo.

\subsection{Caracter\'isticas de los Fabric SDKs}

En general los 4 repos de Fabric SDK existentes siguen las especificaciones de [\cite{fabricsdkspec}], proveyendo las primitivas para acceder a las funcionalidades b\'asicas de la red desde una capa de abstracci\'on que aligera la carga del usuario. Permiten a los desarrolladores interactuar con la blockchain para desplegar e invocar chaincode, escuchar eventos que se puedan generar y recuperar informaci\'on acerca los bloques y trasacciones salvaguardados en el ledger. 

%following paragraph should be supported with examples
El de Go resulta un poco diferente al resto en cuanto a estilo e implementaci\'on; posee un mayor nivel de abstracci\'on que le facilita al programador el proceso de acoplamiento con la plataforma. Los de Java y Node disponen de una capa extra que separa las funciones para el desarrollo de aplicaciones de las administrativas. As\'i, aunque cada uno le aporta las caracter\'isticas propias del lenguaje en particular y le adiciona las modificaciones consideradas prudentes, todos se ajustan a la misma idea central.

% diesnnado en un estilo orientado a objeto. su disenno modular permite a los desarrolladores unir sus implementaciones alternativas para manehar eventos de ocnfirmacion de trasanacciones , evaluacion o query,e tc

Tomemos como ejemplo el de Node.js para un an\'alisis a mayor profundidad. Este est\'a dise\~nado siguiendo el estilo de la programaci\'on orientada a objetos. Se compone de 3 m\'odulos:

\begin{enumerate}
	\item \texttt{fabric-network}: El cual provee APIs de alto nivel para que las aplicaciones clientes interact\'uen con el chaincode, siendo el recomendado para la contrucci\'on de aplicaciones.
	
	\item \texttt{fabric-ca-client}: Que permite la comunicaci\'on con la autoridad certificadora de Fabric.
	
	\item \texttt{fabric-common}: Una API de bajo nivel utilizada para implementar la capacidad de red de fabric-network. Facilita la interacci\'on con los componentes claves de una red de HLF, d\'igase nodos pares, ordenadores y flujos de eventos.
\end{enumerate}

%[esto puede ponerse aqui o pasarlo a la parte de propuesta]
%Escribir aqui un poquito de las notas que hay en la libreta sobre que hace cada repo

El m\'odulo \texttt{fabric-ca-client}, el de mayor inter\'es para este proyecto, es el que provee las capadidades para interactuar con la Fabric CA para el manejo de entidades.  Permite registrar usuarios, inscribirlos para obtener el certificado correspondiente firmado por la Ca,  revocar certificados, revocar usuarios seg\'un su \emph{enrollmentId} o n\'umero identificador de inscripci\'on, entre otros.

Para poder acceder a estas funcionalidades la biblioteca cuenta con la clase 
\\
\texttt{ FabricCAServices} que debe ser instanciada como punto de entrada. Esta, poseedora a su vez de una clase \texttt{FabricCAClient}, expone los m\'etodos para gesti\'on de certificados y entidades plasmados en la segunda. Posee adem\'as una implementaci\'on de la interfaz \texttt{CryptoSuite}, requerida por cada SDK, que encapsula los algoritmos para firmas digitales, encriptaci\'on y hashing seguro. En este caso el repo de node ofrece implementaciones para ECDSA y SHA2/3, aunque admite que se utilicen otras alternativas si son especificadas en las configuraciones de la CrytoSuite.

%Comentar que JAVA es igual orientado a objetos lo que no cuenta con la clase FCAServices

%Puede hablarse de X509 certificate

%Ver en Member Service Provider interface and the COP implementation que definen el enroll y el resto

%Otros detalles ineresantes como concurrencia...
