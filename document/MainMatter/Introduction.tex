\chapter*{Introducción}\label{chapter:introduction}
\addcontentsline{toc}{chapter}{Introducción}
% BibLaTeX + Biber

En los \'ultimos a\~nos se ha producido un incremento acelerado del uso y desarrollo de las \emph{blockchains}. Con m\'as de 900 publicaciones relacionadas hasta inicios del 2019 tan s\'olo en la \emph{Web of Science Core Collection WOC} [\cite{xu2019systematic} , \cite{yli2016current}], resulta sin dudas una tecnolog\'ia que ha irrumpido en el mundo como lo conocemos con intenciones de revolucionarlo. %sin intenciones de quedarse atr\'as.%co intenciones de revolucionarlo
%en la \emph{Web of Science Core Collection WOC}

%Se enfoca como Si bien... a la gente le atrajeron las cryptos... hay un monton de cosas bien cool en este ambito como el hecho de poder desarrollar contratos inteligentes, las apps encima por ello y el monton de aplicaciones q existen.. por lo cual se hace imprescindible en algunos casos como el de HF controlar quien accede a los datos sensibles q se guardan ahi
Si bien la primera propuesta de blockchain conocida data de 1982 [\cite{chaum1979computer}], no es hasta el 2009, con la apararici\'on del \emph{Bitcoin} [\cite{nakamoto2008bitcoin}], que realmente se le empieza a dar importancia. La idea de una moneda digital, independiente de instituciones poco confiables, con la cual fuera posible realizar transacciones an\'onimas pero a la vez cristalinas para todos los usuarios, ciertamente represent\'o, y representa a\'un, una gran competencia ante las industrias bancarias tradicionales. Sin embargo, las potencialidades que esta tecnolog\'ia tiene para ofrecer no se quedan a\'ca. %una alternativa mas atractiva que la de los ... os sistemas tradicionales de pago... las grandes transnacionales bancarias, Ahi se puede poner tambien un parrafito para cerrar diciendo que por ello no fue dificil cautivar al publico, aunque los mas escepticos aun duden.... dispuesta a permitir... que dicha tecnologia (con que cuenta)... no se limitan a estas o este sector

%blockchain data structures harden network security by reducing single-point-of-failure risk, making a database breach difficult.

%not feeling rigth about this sentence

%mencionar otras cryptos Cardano...

Desde la creaci\'on de Ethereum [\cite{ethereum2014ethereum}] como plataforma no s\'olo para criptomonedas, sino como ambiente para la programaci\'on y el despliegue de contratos inteligentes, comenzaron a desarrollarse un sinn\'umero de aplicaciones sobre este tipo de redes.

As\'i, las blockchains como registros digitales distribuidos, p\'ublicos e inmutables han mostrado su val\'ia en m\'ultiples esferas de la vida cotidiana. Se han aplicado a la medicina, la agricultura, el arte, la educaci\'on, la energ\'ia, para la verificaci\'on y el trazado de activos y el manejo de t\'itulos y escrituras [\cite{engelhardt2017hitching} , \cite{abou2019blockchain}].

%Pero, qu\'e sucede con el entorno empresarial?
%Blockchain helps in the verification and traceability of multistep transactions needing verification and traceability. It can provide secure transactions, reduce compliance costs, and speed up data transfer processing. Blockchain technology can help contract management and audit the origin of a product. It also can be used in voting platforms and managing titles and deeds.

%Internet of Things, energy, finance, healthcare, and government

N\'otese que la mayor\'ia de las blockchains conocidas son p\'ublicas o \emph{permissionless}, lo que significa que cualquiera puede interactuar en ellas. Sin embargo, en entornos del tipo empresarial se requiere en muchas ocasiones controlar qui\'enes tienen acceso a la red. Por ejemplo, regulaciones del tipo \emph{Know-Your-Customer KYC} y \emph{Anti-Money Laundering AML} en el ambiente finaciero exigen conocer qui\'en realiza cada transacci\'on. 

A raiz de estos y otros requerimientos funcionales, se han modificado numerosas redes existentes de anta\~no y se han creado otras espec\'ificas al sector empresarial. En este \'utltimo grupo se enmarca \emph{Hyperledger Fabric (HLF)} [\cite{androulaki2018hyperledger}], como una tecnolog\'ia de redes distribu\'idas (del ingl\'es \emph{Distributed Ledger Technology DLT}) dise\~nada para encajar en este panorama.%****Aqui es valido hacer la aclaracion de que DLT no es precisamente blockchain, sino q es mas amplio
%destacar sobresalir dominar

La naturaleza permisionada de HLF permite que los participantes de la red se conozcan entre s\'i, lo cual le aporta un gran nivel de seguridad. 
%A partir de esta informaci\'on los usuarios son capaces de elegir acorde a sus caracter\'isticas y necesidades reales un mecanismo de consenso  a partir de las posibilidades que brinda 
%puedan ser competidores es posibl elegir un mecanismo de consenso acorde a sus necesidades y caracter\'isticas reales.
Posee adem\'as una arquitectua modular y altamente configurable; y constituye una de las mejores plataformas disponibles en t\'erminos de procesamiento de transacciones y latencia de confirmaci\'on de las mismas. %plugging consensus protocol, no necesita de una criptomoneda nativa para incentivar el el minado, lo q a su vez reduce el riesgo de ataques

Una de sus peculiaridades m\'as relevantes, radica en el hecho de ser la primera DLT en brindar soporte para programar contratos inteligentes en lenguajes de prop\'osito general, en lugar de limitar el desarrollo a lenguajes espec\'ificos al dominio (DSL del ingl\'es Domain Specific Language) como lo es Solidity para Ethereum. Su flujo de ejecuci\'on del tipo \emph{execute-order-validate}, en contraste con el tradicional \emph{order-execute}, elimina el no determinismo, puesto que posibilita el filtrado de resultados incorrectos antes de pasar a la fase de ordenamiento. 

%De esta forma, en la actualidad %mas utilizado, sinonimo que ya esta palabra se repite arriba
%que hacen a HF \'unico 

Actualmente, Hyperledger [\cite{hyperledgerorg}] ofrece la capacidad de implementar \emph{smart contracts} en Go, Java y Node.js para interactuar con Hyperledger Fabric. Esto supone una facilidad tremenda para la empresa o individuo que desee trabajar con la DLT. Primero le exime de la necesidad de capacitaci\'on extra, al no requerir el estudio de nuevas herramientas. Segundo, le permite elegir, entre una gama de lenguajes, aquel que a su juicio provea las funcionalidades adecuadas al problema por resolver.%que le acate, que le corresponde %ofrece la posibilidad
% esta bien poner que es Hyperledger el que ofrece eso? Dar una clarification de quien es la empresa ( visto como an open source collaborative effort created to advance the implementation of blockchain technologies).

El Instituto de Criptograf\'ia que radica en la Universidad de La Habana trabaja desde hace alg\'un tiempo en la investigaci\'on y desarrollo de proyectos vinculados a estas l\'ineas. En ese sentido se han efectuado ya los primeros acercamientos para extender la lista de lenguajes mencionados a otros de amplia utilizaci\'on como son Python [\cite{chaincode22python}] y C\# [\cite{chaincode22csharp}].
%radica en la UH? Aclarar esto
%notese que python ya tiene alguito pero de C\Sharp todavia nada, por tanto resulta novedoso

Este trabajo en particular pretende proseguir con el desarrollo de mecanismos que permitan la programaci\'on de contratos inteligentes en C\# para Hyperlegder Fabric.

Para dar continuidad a dicha investigaci\'on, resulta de inter\'es interactuar con el servidor de la autoridad certificadora de Fabric (Fabric-CA), a trav\'es de su REST-API, y as\'i administrar el ciclo de vida de los certificados de usuario. Dichos certificados son la base de la identificaci\'on de individuos mencionada con anterioridad. 

Las plataformas desarrolladas para el intercambio entre cada lenguaje particular y la blockchain de HLF, los Fabric-SDK, cuentan con un componente \emph{fabric-ca-client} que facilita a las aplicaciones registrar nodos y usuarios y establecer identidades confiables en la red. Luego, teniendo en cuenta la hipótesis de que “es viable la concepci\'on de un mecanismo similar a los existentes para extender las capacidades de interacci\'on con la red de Hyperledger Fabric al lenguaje de programaci\'on C\#”, es posible entonces exponer como objetivos
generales de esta tesis: “el dise\~nar e implementar un m\'odulo fabric-ca-client para interactuar con el Fabric-CA empleando C Sharp”.
Con el fin de dar cumplimiento a estos objetivos generales se pueden listar los siguientes objetivos específicos:

%También brinda soporte para envíos de transacciones seudónimas.
 
%como cada vez se hace mas necesario la creacion de aplicaciones que permitan interactuar con el mundo de la blochain y el de afuera. 

\begin{enumerate}
	\item Realizar un estudio de las bibliotecas fabric-sdk ya existentes para Python, Golang, Node.js y Java. Particularmente analizar el m\'odulo fabric-ca-client encargado de cumplir el prop\'osito de este proyecto. 
	
	\item Evaluar qu\'e estructuras y patrones de los ya existentes se pueden llevar a C\#, cu\'ales se pueden reutilizar y qu\'e otros no han sido consideradas a\'un.
	
	\item Concebir, dise\~nar, implementar el m\'odulo fabric-ca-client del fabric-sdk-csharp, para generar identidades criptogr\'aficas con CSharp. Como centro de la creaci\'on brindar soporte para las peticiones de registro, inscripci\'on, renovaci\'on y revocaci\'on de certificados.
	
	%aqui se pueden separar los objetivos en fab-ca-client y fab-ca-server mas c509Cert, pero puede ser excesivo
	
	\item Desarrollar un conjunto de pruebas sobre el m\'odulo implementado, de modo que sea posible comparar objetivamente los resultados con los que reportan otros lenguajes y valorar el cumplimiento de los requerimientos b\'asicos.
\end{enumerate}

El documento en cuesti\'on se estructura de la siguiente forma: El cap\'itulo \ref{chapter:state-of-the-art} realiza un bosquejo de las tecnolog\'ias relevantes para la comprensi\'on del presente estudio. El \ref{chapter:proposal} presenta la propuesta concreta a desarrollar, describe los m\'odulos y sus funcionalidades b\'asicas. El cap\'itulo \ref{chapter:implementation} comenta los resultados obtenidos y las pruebas realizadas. Finalmente, se exponen los \'ultimos an\'alisis y potenciales directivas para encaminar investigaciones futuras.

%Las aplicaciones interactuan con el ledger a traves de un intermediario que seria un sdk, el que a su vez se comunica con el smart contract que es el que realiza las operaciones 'get, put, delete' sobre el world state y registra las transacciones en la blokchain.

%(https://www.bbva.com/es/diferencia-dlt-blockchain/)

%Es habitual pensar que todas las ‘blockchain’ son DLT, es decir, tecnologías de registro distribuido (‘Distributed Ledger Technology’, en inglés), sin embargo,una ‘blockchain’, una cadena de bloques, es un tipo de DLT.Desde un punto de vista más técnico, una DLT es simplemente una base de datos que gestionan varios participantes y no está centralizada. ‘Blockchain’ es una DLT con una serie de características particulares. También es una base de datos —o registro— compartida, pero en este caso mediante unos bloques que, como indica su propio nombre, forman una cadena. Los bloques se cierran con una especie de firma criptográfica llamada ‘hash’; el siguiente bloque se abre con ese ‘hash’, a modo de sello lacrado. De esta forma, se certifica que la información, encriptada, no se ha manipulado ni se puede manipular.
