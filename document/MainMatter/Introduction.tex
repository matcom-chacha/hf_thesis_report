\chapter*{Introducción}\label{chapter:introduction}
\addcontentsline{toc}{chapter}{Introducción}

Descripción: Una plataforma de tipo blockchain empresarial como Hyperledger Fabric requiere que se identifiquen todos los participantes, ya sea un componente de red o un usuario (cliente) que utiliza la plataforma. Esta identificación se implementa a través de certificados digitales, y se necesita una infraestructura para la emisión y gestión de esos certificados.
Si bien se puede usar un tercero para brindar esa infraestructura, Fabric CA proporciona una forma práctica y genera el formato apropiado para Hyperledger Fabric.

El componente fabric-ca-client permite a las aplicaciones registrar nodos y usuarios de aplicaciones para establecer identidades confiables en la red blockchain. También brinda soporte para envíos de transacciones seudónimas.

Problema a resolver: Interactuar con el servidor de la autoridad certificadora de Fabric (Fabric-CA) a través de su REST-API, para administrar el ciclo de vida de los certificados de usuario, como el registro, la inscripción, la renovación y la revocación.

Objetivo general: Dise\~nar e implementar un módulo fabric-ca-client para interactuar con el Fabric-CA empleando C Sharp


