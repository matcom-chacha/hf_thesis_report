\chapter*{Introducción}\label{chapter:introduction}
\addcontentsline{toc}{chapter}{Introducción}
No debe excederse de 10 cuartillas
•Contexto histórico/social donde se desarrolla la 
investigación.
•Antecedentes del problema científico
• Breve presentación de la problemática
• Actualidad
• Novedad científica 
• Importancia teórica y práctica 
• Diseño teórico: 
Problema científico
Objeto de estudio 
Objetivos 
Campo de acción
Hipótesis o Pregunta científica
• Estructuración del trabajo

 Designa una dificultad que no puede resolverse
 automáticamente, sino que requiere una
 investigación conceptual o empírica.
 • Es la interrogante que nos planteamos en
 términos científicos, a la cual le vamos a buscar
 solución.
 • Debe tener una adecuada formulación.
 • Generalmente se formula como pregunta.
 
 lternativa de respuesta al problema
 científico.
 •Es una suposición, una respuesta anticipada al
 problema científico.
 •Es una solución tentativa al mismo.
 •Constituye un elemento orientador en el
 proceso científico.
 •Generalmente se formula como afirmación.
 
 •Su formulación debe involucrar resultados
 concretos del trabajo.
 •Deben plantearse mediante el infinitivo de los
 verbos.
 •Plantear objetivos generales y objetivos específicos.
 •Los objetivos generales deben referirse a resultados
 amplios
 •Los objetivos específicos hacen mención a
 situaciones específicas o particulares.
 
Descripción: Una plataforma de tipo blockchain empresarial como Hyperledger Fabric requiere que se identifiquen todos los participantes, ya sea un componente de red o un usuario (cliente) que utiliza la plataforma. Esta identificación se implementa a través de certificados digitales, y se necesita una infraestructura para la emisión y gestión de esos certificados.
Si bien se puede usar un tercero para brindar esa infraestructura, Fabric CA proporciona una forma práctica y genera el formato apropiado para Hyperledger Fabric.

El componente fabric-ca-client permite a las aplicaciones registrar nodos y usuarios de aplicaciones para establecer identidades confiables en la red blockchain. También brinda soporte para envíos de transacciones seudónimas.

Problema a resolver: Interactuar con el servidor de la autoridad certificadora de Fabric (Fabric-CA) a través de su REST-API, para administrar el ciclo de vida de los certificados de usuario, como el registro, la inscripción, la renovación y la revocación.

Objetivo general: Dise\~nar e implementar un módulo fabric-ca-client para interactuar con el Fabric-CA empleando C Sharp


