\chapter{Detalles de Implementación y Experimentos}\label{chapter:implementation}

•Metodología empleada
•Resultados obtenidos y su análisis, 
valoración e interpretación

Por capítulos 
y epígrafes(opcional)

•Síntesis de los resultados alcanzados en la
revisión de la bibliografía relacionada con el
tema.
•Presentación organizada de los
conocimientos científicos acumulados sobre
el tema.
•Presentación y análisis de investigaciones
anteriores.
•Presupuestos o antecedentes de que se
parte.
•Posición que asume el autor sobre el tema.

Metodología empleada
•Marco conceptual
•Tareas realizadas
•Métodos empleados y tipos de 
proyecto.
•Análisis de los resultados obtenidos

El proyecto de desarrollo tecnológico
El sello distintivo de este tipo de proyecto es que se orienta hacia
la obtención de productos tangibles: un medio diagnóstico, un
dispositivo para la realización de biopsias, un software, un
modelo para la predicción del rendimiento académico,
maquetas, modelos experimentales.
La obtención del producto se acompaña de la evaluación de
sus propiedades.
El proyecto de investigación
El rasgo que tipifica al proyecto de investigación es la existencia de
una intención cognoscitiva que prevalece sobre cualquier otro
propósito en el proyecto. Conocer quiere decir arribar a
proposiciones verdaderas o más completas sobre un objeto de
estudio y/o generar, confirmar, refutar o verificar hipótesis en
relación con dicho objeto.
El QUÉ el POR QUÉ, el PARA QUÉ y el CÓMO figuran como
componentes constantes en los textos en que se materializa todo
proyecto.

Selección de los métodos y técnicas
Debe responder a:
• Tipo de objetivos e hipótesis.
• Naturaleza del objeto de estudio.
•Ventajas y limitaciones de los métodos y 
técnicas.
• Características de la institución en que se 
realiza la investigación.
•Recursos materiales y humanos utilizados
