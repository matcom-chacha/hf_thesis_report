\begin{conclusions}
 El presente trabajo proporciona los primeros acercamientos para el dise\~no e implementaci\'on de un SDK que facilite el intercambio entre una red de Hyperledger Fabric y una aplicaci\'on cualquiera en el lenguaje de programaci\'on C\#. 
 
 Las ventajas de HLF como blockchain empresarial son innegables, por lo que resulta de inter\'es extender la lista de lenguajes a los que provee soporte.
 
 El documento cuenta con una breve rese\~na de las caractr\'isticas generales de la blockchain HLF, resaltando las ventajas que podr\'ian ofrecerse al extender dicha lista. Igualmente se presentaron los mecanismos de autenticaci\'on y autorizaci\'on en esta plataforma, los cuales constituyen la base de las funcionalidades a implementar.

 Para comprender el funcionamiento a replicar, fueron analizadas las bibliotecas \texttt{fabric-sdk} ya existentes para Python, Golang, Node.js y Java. Se evaluaron las estructuras y patrones utilizados, descartando algunos y conservando los m\'as adecuados para el lenguaje en cuesti\'on.
 
 A partir de los estudios realizados, fue posible dise\~nar e implementar el m\'odulo \texttt{FabricCaClient} de un  \texttt{fabric-sdk-csharp}, para generar identidades criptogr\'aficas con C\#.
 
 Para garantizar el soporte para las peticiones b\'asicas de registro, inscripci\'on, renovaci\'on y revocaci\'on de certificados fue necesario tambi\'en implementar funcionalidades para la creaci\'on de llaves criptogr\'aficas y generaci\'on de firmas digitales. estos aspectos fueron recogidos en la interfaz \emph{CryptoSuite} y la clase \emph{CryptoPrimitives}. 
 
 Por \'ultimo, como forma de garantizar la persistencia de datos, se implement\'o un wallet o billetera que permita a las aplicaciones almacenar y acceder a información sobre identidades de usuarios.
 
 El c\'odigo escrito fue objeto de pruebas, verificando que los resultados obtenidos se adecuaran al comportamiento esperado. Como resultado se arriv\'o a un un código robusto, flexible, libre de errores, capaz de satisfacer las
 funcionalidades básicas para las que fue concebido.
 
\end{conclusions}
