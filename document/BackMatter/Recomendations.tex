\begin{recomendations}
Como se ha comentado, la implementaci\'on propuesta no constituye m\'as que un primer acercamiento para a\~nadir soporte al desarrollo de aplicaciones sobre Hyperledger Fabric utilizando C\#. Teniendo esto es cuenta, es posible indicar un conjunto de l\'ineas a seguir en funci\'on de acercar el trabajo futuro hacia al objetivo propuesto:

\begin{enumerate}
	%lo del user
	%extender test para cubirs otros comportamientos que pueden haberlse saltado
	\item Refactorizar el c\'odigo para encapsular ciertos comportamientos en clases espec\'ificas, como los par\'ametros requeridos en cada intecambio con la CA. Esto facilitar\'ia el control de errores y ajustar\'ia m\'as el al paradigma de orientaci\'on a objetos del lenguaje utilizado.
	
	\item Ampliar las funcionalidades de la clase \emph{CriptoPrimitives}, mediante la inclusi\'on de los m\'etodos de \emph{Encrypt}, \emph{Decrypt} y \emph{Verify}.
	
	\item Extender la clase \emph{WalletStore} e implementar la clase \emph{HsmX509Identity} para brindar a las apliciones un medio m\'as seguro donde salvaguardar su material criptogr\'afico.
\end{enumerate}

Las funcionalidades recomendadas responden fudamentalmente al trabajo efectuado hasta el momento. En el futuro se podr\'ia proseguir con otros elementos del m\'odulo \texttt{FabricNetwork} y \texttt{FabricCommon} que permitan conectarse a una red de Fabric, acceder a una canal e instanciar y consultar chaincodes.

\end{recomendations}
