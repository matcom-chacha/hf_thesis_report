\documentclass[12pt,oneside]{uhthesis}
\usepackage{subfigure}
%\usepackage[ruled,lined,linesnumbered,titlenumbered,algochapter,spanish,onelanguage]{algorithm2e}
\usepackage{amsmath}
\usepackage{amssymb}
\usepackage{amsbsy}
\usepackage{caption,booktabs}
\captionsetup{ justification = centering }
%\usepackage{mathpazo}
\usepackage{float}
\setlength{\marginparwidth}{2cm}
\usepackage{todonotes}
\usepackage{listings}
\usepackage{xcolor}
\usepackage{multicol}
\usepackage{graphicx}
\usepackage{caption}
%\captionsetup[table]{position=bottom} 
\floatstyle{plaintop}
%\restylefloat{table}
\addbibresource{Bibliography.bib}
% \setlength{\parskip}{\baselineskip}%
\renewcommand{\tablename}{Tabla}
%\renewcommand{\listalgorithmcfname}{Índice de Algoritmos}
%\dontprintsemicolon
%\SetAlgoNoEnd

\definecolor{codegreen}{rgb}{0,0.6,0}
\definecolor{codegray}{rgb}{0.5,0.5,0.5}
\definecolor{codepurple}{rgb}{0.58,0,0.82}
\definecolor{backcolour}{rgb}{0.95,0.95,0.92}
\definecolor{redstrings}{rgb}{0.9,0,0}
\lstdefinestyle{mystyle}{
	language=[Sharp]C,
	backgroundcolor=\color{white},   
	commentstyle=\color{codegreen},
	keywordstyle=\color{blue},
	morekeywords={partial, var, value, get, set, T},
	numberstyle=\tiny\color{codegray},
	stringstyle=\color{redstrings},
	%    basicstyle=\ttfamily\footnotesize,
	breakatwhitespace=true,         
	breaklines=true,                 
	captionpos=b,                    
	keepspaces=true,                 
	numbers=left,                    
	numbersep=5pt,                  
	showspaces=false,                
	showstringspaces=false,
	showtabs=false,                  
	tabsize=4,
	%    
	frame=lines,
	escapeinside={(*@}{@*)},
	%commentstyle=\color{greencomments},
	basicstyle=\ttfamily,
}

\lstset{style=mystyle}

%\lstdefinestyle{mystyle}{
%    backgroundcolor=\color{backcolour},   
%    commentstyle=\color{codegreen},
%    keywordstyle=\color{purple},
%    numberstyle=\tiny\color{codegray},
%    stringstyle=\color{codepurple},
%    basicstyle=\ttfamily\footnotesize,
%    breakatwhitespace=false,         
%   breaklines=true,                 
%   captionpos=b,                    
%    keepspaces=true,                 
%   numbers=left,                    
%    numbersep=5pt,                  
%    showspaces=false,                
%   showstringspaces=false,
%    showtabs=false,                  
%   tabsize=4
%}

%\lstset{style=mystyle}

\title{Módulo fabric-ca-client \\ del SDK de Fabric \\ para gestionar identidades criptográficas \\ con C\#.}
\author{\\\vspace{0.25cm}Gabriela B. Mart\'inez Giraldo}
%\advisor{\\\vspace{0.25cm}Ing. Daniel Mena \\\vspace{0.25cm}Ing. Camilo  Denis González\\\vspace{0.2cm}Dr. Miguel Katrib Mora}

\advisor{\\\vspace{0.25cm}Camilo Denis González
	\\\vspace{0.25cm}Daniel Mena Frias
\\\vspace{0.25cm}Miguel Katrib Mora}
\degree{Licenciado en Ciencia de la Computación}
\faculty{Facultad de Matemática y Computación}

\date{Enero del 2023\\\vspace{0.25cm}\href{https://github.com/ic-matcom/fabric-sdk-csharp}{github.com/ic-matcom/fabric-sdk-csharp}}

\logo{Graphics/uhlogo}

\makenomenclature

\renewcommand{\vec}[1]{\boldsymbol{#1}}
\newcommand{\diff}[1]{\ensuremath{\mathrm{d}#1}}
\newcommand{\me}[1]{\mathrm{e}^{#1}}
\newcommand{\pf}{\mathfrak{p}}
\newcommand{\qf}{\mathfrak{q}}
%\newcommand{\kf}{\mathfrak{k}}
\newcommand{\kt}{\mathtt{k}}
\newcommand{\mf}{\mathfrak{m}}
\newcommand{\hf}{\mathfrak{h}}
\newcommand{\fac}{\mathrm{fac}}
\newcommand{\maxx}[1]{\max\left\{ #1 \right\} }
\newcommand{\minn}[1]{\min\left\{ #1 \right\} }
\newcommand{\lldpcf}{1.25}
\newcommand{\nnorm}[1]{\left\lvert #1 \right\rvert }
\renewcommand{\lstlistingname}{Ejemplo de código}
\renewcommand{\lstlistlistingname}{Ejemplos de código}

\begin{document}

\frontmatter

\maketitle
%\begin{center}
	%\includegraphics[width=2.5cm]
	%{Graphics/qr.png}
	%\date{Fecha}
	%\\\vspace{0.4cm}
%	20 de enero de 2023
%\end{center}

\begin{dedication}
    \emph{“En la profundidad del invierno, aprendí que en mi interior hay un verano invencible.”}
    
    Albert Camus
\end{dedication}
\begin{acknowledgements}
	A mi mam\'a, la raz\'on por la que me levanto cada d\'ia. Mi hero\'ina.
	
	A mi pap\'a que ha sido siempre el mejor ejemplo de fortaleza y energ\'ia de vida.
	
	A este famili\'on inmenso que ha estado siempre de cerquita velando por m\'i, ese que no se deja vencer por muchos obst\'aculos que la vida disponga en el camino. A mis t\'ias Ara, Deisy y mi t\'io Alberto. A mis primos Ale, Franci, Yanila, Mabel y H\'ector. Gracias por el cari\~no, por celebrar mis logros como suyos, por impulsarme siempre para seguir adelante.
	
	A los que ya no est\'an, pero s\'e que desde el cielo velan por nosotros. A mis abuelas Chela y Cata y mis t\'ios Daisy y Hugo.
	
	A Betsy, Tanu, Claudia, Tedy y Ramses, mis hermanos, que a pesar de no habernos criado juntos han sentido suyas mis m\'as duras experiencias.
	
	A Ami, Sandri y Ariel por compartir conmigo alegr\'ias, l\'agrimas y locuras. A Yosy, Susana y Ana Laura por ser junto a Ami las hermanas que la Lenin me brind\'o. A Vitico por ser mi amigo m\'as viejo e incondicional. A todos por darme fuerzas cuando no ten\'ia, por nunca dejarme caer.
	
	A Nadia, Luis y Jose por las horas de oficina que se hicieron tan amenas.
	
	Al resto de esos amigos que la universidad me ha dado, lo m\'as bonito que me llevo de esta etapa de vida. A Carmen, Karla, Amanda, Aldo, Mederos, Osmany, Labu, Guaty, Kike, Pirlo, Henry, Javier, Laura, Sheyla, Thalia, Carlitos, Daniel, los Rodros y el team Santa F\'e.
	
	A mi tutor Camilo por no dudar dos veces cuando ped\'i su ayuda; por no preguntar razones siquiera antes de lanzarse a mi gu\'ia.
	
	A mis tutores Daniel, Katrib por el tiempo dedicado. Al profesor Luis Ramiro y al resto del departamento por su disposici\'on de ayudar. 
	
	A la profe Carmen por la confianza despositada en m\'i. Por mover cielo y tierra para hacer esta tesis una realidad.
	
	Al profesor Luciano por su tiempo y comprensi\'on. A la profe Lucina por tener siempre una palabra amorosa a la mano.
	
	Al resto de los docentes que hicieron estos a\~nos de carrera los m\'as maravillosos. Gracias por ense\~narnos m\'as que las materias b\'asicas, por la entrega en todo lo que hacen, por empujarnos a seguir adelante y no rendirnos jam\'as. A Juan Pablo, Wilfredo, Celia, Claudia, Camila, Fernando, Somoza, Piad y otros tantos que nos ense\~naron a so\~nar.
	
	Somos un pedacito de todos los que nos rodea. Es s\'olo con la compa\~nia de gente tan maravilosa que la vida toma color.
\end{acknowledgements}
\begin{opinion}
    El trabajo Módulo fabric-ca-client del SDK de Fabric para gestionar identidades criptogr\'aficas con C\# desarrollado por la estudiante Gabriela B. Martínez Giraldo, cumple con los requisitos para la culminación de la carrera de Ciencia de la Computación de la Universidad de La Habana.
   
    Este módulo es un componente de un SDK implementado en el lenguaje de programación C\# para interactuar con la tecnología de blockchain Hyperledger Fabric (HF). Siendo Hyperledger Fabric una de las plataformas usadas para el desarrollo de nuestras aplicaciones blockchain y siendo una debilidad la ausencia de bibliotecas, SDK y API en C\# hacen pertinente y útil el desarrollo de este módulo.
    
    Es válido aclarar que la diplomante ha mostrado excelentes habilidades técnicas durante el desarrollo del trabajo, demostrando interés, dominio del tema y cumpliendo con todos los requisitos definidos por el cliente. Para ello, comenzó con la asimilación y estudio de las tecnologías indicadas por los tutores, mostrando además buenas capacidades de asimilación e independencia
    
    Por tanto, felicitamos a la estudiante por la labor desarrollada y consideramos que la tesis reúne los estándares metodológicos exigidos por la Facultad de Matemática y Computación de la Universidad de la Habana, para ser presentada y sometida a evaluación en su ejercicio de defensa.
    
    Exhortamos a la estudiante a trabajar para convertir este trabajo en una publicación.
    
    La Habana, enero del 2022\\
    
    Camilo Denis González $\_\_\_\_\_\_\_\_\_\_\_\_\_\_\_$\\
    
    Camilo Denis González $\_\_\_\_\_\_\_\_\_\_\_\_\_\_\_$\\
    
    Miguel Katrib Mora $\_\_\_\_\_\_\_\_\_\_\_\_\_\_\_\_\_$\\
\end{opinion}
\begin{resumen}
\emph{Hyperledger Fabric (HLF)} constituye una tecnolog\'ia de redes distribuidas (DLT) dise\~nada espec\'ificamente para el sector empresarial. 
%Resulta una de las mejores plataformas disponibles en t\'erminos de procesamiento de transacciones y latencia de confirmaci\'on de las mismas; y es ampliamente reconocida por su arquitectura modular y altamente configurable.

Una de sus peculiaridades m\'as relevantes, radica en el hecho de ser la primera DLT en brindar soporte para programar contratos inteligentes en lenguajes de prop\'osito general, en lugar de limitar el desarrollo a lenguajes espec\'ificos al dominio (DSL). Actualmente, ofrece la capacidad de implementar contratos inteligentes en Go, Java y Node.js.

Por las amplias ventajas que esta plataforma reporta, resulta de inter\'es extender el conjunto de lenguajes a los que ofrece soporte.

El presente trabajo propone el diseño e implementaci\'on de un m\'odulo \emph{fabric-ca-client} de un \emph{Fabric-SDK} en C\#, con el fin de poder interactuar con la Fabric CA de HLF utilizando este lenguaje. 
%Este proyecto forma parte de uno m\'as ambicioso que aspira a poder desarrollar chaincode en C\#.
\end{resumen}

\begin{abstract}
	\emph{Hyperledger Fabric} (HLF) is a distributed networking technology (DLT)
	designed specifically for the business sector.
	
	One of its most relevant peculiarities lies in the fact that it is the first DLT
	in providing support for programming smart contracts in general purpose languages, rather than limiting development to domain-specific ones(DSL).
	
	It currently offers the ability to implement smart contracts in Go, Java, and Node.js.
	
	Due to the wide advantages this platform reports, it results of interest to extend the set of languages it supports.
	
	This paper proposes the design and implementation ot a \texttt{fabric-ca-client} module of a \emph{Fabric-SDK} written in C\#, in order to interact with a HLF's CA usign this language.
\end{abstract}


\include{FrontMatter/Contents}

\mainmatter

\chapter*{Introducción}\label{chapter:introduction}
\addcontentsline{toc}{chapter}{Introducción}
% BibLaTeX + Biber

En los \'ultimos a\~nos se ha producido un incremento acelerado del uso y desarrollo de las \emph{blockchains}. Con m\'as de 900 publicaciones relacionadas hasta inicios del 2019 tan s\'olo en la \emph{Web of Science Core Collection (WOS)} [\cite{xu2019systematic} , \cite{yli2016current}], resulta sin dudas una tecnolog\'ia que ha irrumpido en el mundo como lo conocemos con intenciones de revolucionarlo. %sin intenciones de quedarse atr\'as.%co intenciones de revolucionarlo
%en la \emph{Web of Science Core Collection WOC}

\nomenclature[wos]{\textbf{Web of Science Core Collection (WOS)}}{Base de datos de citas l\'ider en el mundo. Contiene registros de art\'iculos de las revistas de mayor impacto, incluidas publicaciones libres, actas de congresos y libros.}

%Se enfoca como Si bien... a la gente le atrajeron las cryptos... hay un monton de cosas bien cool en este ambito como el hecho de poder desarrollar contratos inteligentes, las apps encima por ello y el monton de aplicaciones q existen.. por lo cual se hace imprescindible en algunos casos como el de HF controlar quien accede a los datos sensibles q se guardan ahi
Si bien la primera propuesta de blockchain conocida data de 1982 [\cite{chaum1979computer}], no es hasta el 2009, con la aparici\'on del \emph{Bitcoin} [\cite{nakamoto2008bitcoin}], que comienza a cobrar importancia. La idea de una moneda digital, independiente de instituciones poco confiables, con la cual fuera posible realizar transacciones casi an\'onimas, pero a la vez transparentes a todos los usuarios, ciertamente represent\'o, y representa a\'un, una gran competencia ante las industrias bancarias tradicionales. Sin embargo, las potencialidades que esta tecnolog\'ia tiene para ofrecer no se quedan ac\'a. %una alternativa mas atractiva que la de los ... os sistemas tradicionales de pago... las grandes transnacionales bancarias, Ahi se puede poner tambien un parrafito para cerrar diciendo que por ello no fue dificil cautivar al publico, aunque los mas escepticos aun duden.... dispuesta a permitir... que dicha tecnologia (con que cuenta)... no se limitan a estas o este sector

%blockchain data structures harden network security by reducing single-point-of-failure risk, making a database breach difficult.

%not feeling rigth about this sentence

%mencionar otras cryptos Cardano...

Desde la creaci\'on de Ethereum [\cite{ethereum2014ethereum}] como plataforma no s\'olo para criptomonedas, sino como ambiente para la programaci\'on y el despliegue de contratos inteligentes, comenzaron a desarrollarse un sinn\'umero de aplicaciones sobre este tipo de redes.

As\'i, las blockchains como registros digitales distribuidos, p\'ublicos e inmutables han mostrado su val\'ia en m\'ultiples esferas de la vida cotidiana. Se han aplicado a la medicina, la agricultura, el arte, la educaci\'on, la energ\'ia, para la verificaci\'on y el trazado de activos y el manejo de t\'itulos y escrituras [\cite{abou2019blockchain}, \cite{engelhardt2017hitching}].

%Pero, qu\'e sucede con el entorno empresarial?
%Blockchain helps in the verification and traceability of multistep transactions needing verification and traceability. It can provide secure transactions, reduce compliance costs, and speed up data transfer processing. Blockchain technology can help contract management and audit the origin of a product. It also can be used in voting platforms and managing titles and deeds.

%Internet of Things, energy, finance, healthcare, and government

N\'otese que la mayor\'ia de las blockchains conocidas son p\'ublicas o \emph{permissionless}, lo que significa que cualquiera puede interactuar en ellas. Sin embargo, en entornos del tipo empresarial se requiere en muchas ocasiones controlar qui\'enes tienen acceso a la red. Por ejemplo, regulaciones del tipo \emph{Know-Your-Customer (KYC)} y \emph{Anti-Money Laundering (AML)} en el ambiente financiero exigen conocer qui\'en realiza cada transacci\'on. 

A ra\'iz de estos y otros requerimientos funcionales, se han modificado numerosas redes existentes de anta\~no y se han creado otras espec\'ificas al sector empresarial. En este \'utltimo grupo se enmarca \emph{Hyperledger Fabric (HLF)} [\cite{androulaki2018hyperledger}], como una tecnolog\'ia de redes distribuidas (del ingl\'es \emph{Distributed Ledger Technology DLT}) dise\~nada para encajar en este panorama.%****Aqui es valido hacer la aclaracion de que DLT no es precisamente blockchain, sino q es mas amplio
%destacar sobresalir dominar

La naturaleza permisionada de HLF permite que los participantes de la red se conozcan entre s\'i, lo cual le aporta un cierto nivel de confianza.
%gran nivel de seguridad. 
%A partir de esta informaci\'on los usuarios son capaces de elegir acorde a sus caracter\'isticas y necesidades reales un mecanismo de consenso  a partir de las posibilidades que brinda 
%puedan ser competidores es posibl elegir un mecanismo de consenso acorde a sus necesidades y caracter\'isticas reales.
Posee adem\'as una arquitectura modular y altamente configurable; y constituye una de las mejores plataformas disponibles en t\'erminos de procesamiento de transacciones y latencia de confirmaci\'on de las mismas [\cite{hlfdocs}]. %plugging consensus protocol, no necesita de una criptomoneda nativa para incentivar el el minado, lo q a su vez reduce el riesgo de ataques

Una de sus peculiaridades m\'as relevantes, radica en el hecho de ser la primera DLT en brindar soporte para programar contratos inteligentes (\emph{smart contracts}) en lenguajes de prop\'osito general, en lugar de limitar el desarrollo a lenguajes espec\'ificos al dominio (DSL del ingl\'es Domain Specific Language), como lo es Solidity para Ethereum. %Su flujo de ejecuci\'on del tipo \emph{execute-order-validate}, en contraste con el tradicional \emph{order-execute}, elimina el no determinismo, puesto que posibilita el filtrado de resultados incorrectos antes de pasar a la fase de ordenamiento. 

%De esta forma, en la actualidad %mas utilizado, sinonimo que ya esta palabra se repite arriba
%que hacen a HF \'unico 

Actualmente, Hyperledger [\cite{hyperledgerorg}] ofrece la capacidad de implementar smart contracts en Go, Java y Node.js para interactuar con Hyperledger Fabric. Esto supone una facilidad tremenda para la empresa o individuo que desee trabajar con la DLT. Primero le exime de la necesidad de capacitaci\'on extra, al no requerir el estudio de nuevas herramientas. Segundo, le permite elegir, entre una gama de lenguajes, aquel que a su juicio provea las funcionalidades adecuadas al problema por resolver.%que le acate, que le corresponde %ofrece la posibilidad
% esta bien poner que es Hyperledger el que ofrece eso? Dar una clarification de quien es la empresa ( visto como an open source collaborative effort created to advance the implementation of blockchain technologies).

% grupo Blockchain del Instituto de Criptografía de la Facultad de Matemática y Computación de la Universidad de La Habana

El Instituto de Criptograf\'ia de la Universidad de La Habana trabaja desde hace alg\'un tiempo en el an\'alisis y desarrollo de proyectos vinculados a estas l\'ineas de investigaci\'on. En ese sentido se han efectuado ya los primeros acercamientos ([\cite{chaincode22python}, \cite{chaincode22csharp}]) para extender la lista de lenguajes mencionados a otros de amplia utilizaci\'on como son Python y C\#.
%radica en la UH? Aclarar esto
%notese que python ya tiene alguito pero de C\Sharp todavia nada, por tanto resulta novedoso

Este trabajo en particular, pretende proseguir con el desarrollo de mecanismos que permitan la programaci\'on de contratos inteligentes en C\# para Hyperlegder Fabric.

Para dar continuidad a dicha investigaci\'on, resulta de inter\'es interactuar con el servidor de la autoridad certificadora de Fabric (Fabric-CA), a trav\'es de su REST-API, y as\'i administrar el ciclo de vida de los certificados de usuario. Dichos certificados son la base de la identificaci\'on de individuos mencionada con anterioridad. 

Los Fabric SDK son las plataformas desarrolladas para el intercambio entre cada lenguaje particular y la blockchain de HLF. Su existencia es la que garantiza el soporte de la variada gama de lenguajes que hoy en d\'ia pueden interactuar con esta plataforma. Constituyen bibliotecas que proporcionan un conjunto integral de APIs para administrar recursos en Fabric, proporcionando una interfaz uniforme que facilita el desarrollo y la implementación de aplicaciones sobre esta. 

Los Fabric-SDK, cuentan con un componente \emph{fabric-ca-client} que permite a las aplicaciones registrar nodos y usuarios, y establecer identidades confiables en la red. Luego, tras considerar lo anterior, es posible entonces exponer como objetivos
generales de esta tesis: “Diseñar e implementar el m\'odulo \texttt{fabric-ca-client} como m\'odulo inicial de un Fabric-SDK escrito en C\#, con el fin de poder interactuar con la Fabric CA de HLF utilizando el lenguaje mencionado”.

%la hipótesis de que “es viable la concepci\'on de un mecanismo similar a los existentes para extender las capacidades de interacci\'on con la red de Hyperledger Fabric al lenguaje de programaci\'on C\#”, es posible entonces exponer como objetivos generales de esta tesis: “Diseñar e implementar el m\'odulo \texttt{fabric-ca-client} como m\'odulo inicial de un Fabric-SDK escrito en C\#, con el fin de poder interactuar con la Fabric CA de HLF utilizando el lenguaje mencionado”.

%“dise\~nar e implementar un m\'odulo \texttt{fabric-ca-client} para interactuar con el Fabric-CA empleando C Sharp”.

En orden de dar cumplimiento a estos objetivos generales se pueden listar los siguientes objetivos específicos:

%También brinda soporte para envíos de transacciones seudónimas.
 
%como cada vez se hace mas necesario la creacion de aplicaciones que permitan interactuar con el mundo de la blochain y el de afuera. 

\begin{enumerate}
	\item Realizar un estudio de las bibliotecas \texttt{fabric-sdk} ya existentes para Python, Golang, Node.js y Java. Particularmente analizar el m\'odulo \texttt{fabric-ca-client} encargado de cumplir el prop\'osito de este proyecto. 
	
	\item Evaluar qu\'e estructuras y patrones de los ya existentes se pueden llevar a C\#, cu\'ales se pueden reutilizar y qu\'e otros no han sido considerados a\'un.
	
	\item Concebir, dise\~nar, implementar el m\'odulo \texttt{fabric-ca-client} del
	\\
	\texttt{fabric-sdk-csharp}, para generar identidades criptogr\'aficas con C\#. Garantizar un soporte para las peticiones b\'asicas de registro, inscripci\'on, renovaci\'on y revocaci\'on de certificados.
	
	%aqui se pueden separar los objetivos en fab-ca-client y fab-ca-server mas c509Cert, pero puede ser excesivo
	
	\item Desarrollar un conjunto de pruebas sobre el m\'odulo implementado, de modo que sea posible comparar objetivamente los resultados con los que reportan otros lenguajes y valorar el cumplimiento de los requerimientos b\'asicos.
\end{enumerate}

El documento en cuesti\'on se estructura de la siguiente forma: En el cap\'itulo \ref{chapter:state-of-the-art} se realiza un bosquejo de las tecnolog\'ias relevantes para la comprensi\'on del presente estudio. El \ref{chapter:proposal} presenta la propuesta concreta a desarrollar, describe los m\'odulos y sus funcionalidades b\'asicas. El cap\'itulo \ref{chapter:implementation} comenta los resultados obtenidos y las pruebas realizadas. Finalmente, se exponen los \'ultimos an\'alisis y potenciales directivas para encaminar investigaciones futuras.

%Las aplicaciones interactuan con el ledger a traves de un intermediario que seria un sdk, el que a su vez se comunica con el smart contract que es el que realiza las operaciones 'get, put, delete' sobre el world state y registra las transacciones en la blokchain.

%(https://www.bbva.com/es/diferencia-dlt-blockchain/)

%Es habitual pensar que todas las ‘blockchain’ son DLT, es decir, tecnologías de registro distribuido (‘Distributed Ledger Technology’, en inglés), sin embargo,una ‘blockchain’, una cadena de bloques, es un tipo de DLT.Desde un punto de vista más técnico, una DLT es simplemente una base de datos que gestionan varios participantes y no está centralizada. ‘Blockchain’ es una DLT con una serie de características particulares. También es una base de datos —o registro— compartida, pero en este caso mediante unos bloques que, como indica su propio nombre, forman una cadena. Los bloques se cierran con una especie de firma criptográfica llamada ‘hash’; el siguiente bloque se abre con ese ‘hash’, a modo de sello lacrado. De esta forma, se certifica que la información, encriptada, no se ha manipulado ni se puede manipular.

\chapter{Antecedentes}\label{chapter:state-of-the-art}

En el presente cap\'itulo se ofrecen una serie de conceptos relevantes al tema en cuesti\'on. Su prop\'osito es el de facilitar la comprensi\'on de las bases en las que se enmarca esta investigaci\'on. %este proyecto

La mayor\'ia de las definiciones que se proveen tienen como fuente al \emph{paper} de Hyperledger Fabric [\cite{androulaki2018hyperledger}], la documentaci\'on oficial de esta blockchain [\cite{hyperledgerorg}] y la de Fabric CA [\cite{fabricca}]. Para el resto se esclarecen las referencias apropiadas.%conceptos, se presentan, definiciones provienen de % adecuadas % debidamente referenciados %referencias correspondientes

\section{\textquestiondown Qu\'e se entiende por Blockchain?}

Una blockchain o cadena de bloques no es m\'as que un registro\footnote{com\'unmente conocido como \emph{ledger} o libro mayor de contabilidad} inmutable y distribuido entre una red de nodos. Cada nodo mantiene una copia del ledger, la cual se actualiza por medio de transacciones ejecutadas y validadas a partir de un protocolo de consenso. La informaci\'on (transacciones) se agrupa en bloques que incluyen un hash del bloque precedente, de forma que resulta altamente complejo alterar los datos de una transacci\'on sin alterar tambi\'en la de los elementos siguientes.

%libro mayor de transacciones falsificar

Entre las ventajas de esta tecnolog\'ia es posible destacar:

\begin{enumerate}
	\item  No censura: Una de las caracter\'isticas m\'as relevantes de las plataformas blockchains es que en ellas no suele existir una autoridad centralizada con total control sobre la red. Esto es posible gracias a los mecanismos de consenso como \emph{Proof or Work} o \emph{Proof of Stake}, que aseguran se llegue a un acuerdo respecto al estado del ledger entre todos los nodos.
	
	
	\nomenclature[pos]{\textbf{Proof of Stake (POS)}}{La prueba de participación es un protocolo utilizado en blockchain que permite a los usuarios validar transacciones manteniendo una cierta cantidad de criptomonedas. Este protocolo permite a los usuarios obtener recompensas por tener criptomonedas, lo que ayuda a mantener la red en pie. Fue creado para reemplazar al conocido Proof of Work, aportando una mejor seguridad y escalabilidad a las redes que lo implementen.}
	
	\nomenclature[pow]{\textbf{Proof of Work (POW)}}{La prueba de trabajo es un mecanismo de consenso descentralizado que requiere que los miembros de la red se esfuercen en resolver un número hexadecimal encriptado. La prueba de trabajo también se denomina minería, en referencia a recibir una recompensa por el trabajo realizado.}
		
	De esta forma, se garantiza una especie de confianza entre elementos desconocidos de la red y se establece un protocolo que impide que una \'unica parte sea capaz de censurar al resto.
	
	%Es v\'alido aclarar que en una plataforma altamente configurable como lo es HLF s\'i puede darse el caso en el que el poder de decisi\'on se concentre en unos pocos elementos de la red.
	
	%que e nodo q se annade sea la unica version de la readlidad a partir de lo acordado por todos los nodos en la blockchain
		
	\item Seguridad: A diferencia de lo que es costumbre en las bases de datos tradicionales, la informaci\'on que se almacena en una blockchain es persistente e inmutable. Gracias al mecanismo de enlazar bloques a partir de su hash se hace muy complejo que un atacante manipule los datos, pues, como se mencionaba anteriormente, resaltar\'ia la falla en los bloques consecutivos. Adicionalmente, dado que la informaci\'on no se respalda en un \'unico destino, sino que se distribuye por todos los nodos de la red, las acciones fraudulentas de este tipo se encuentran un escenario a\'un m\'as complejo de penetrar.
	
	%con las operaciones CRUD (funciones para crear, leer, actualizar y borrar)
	
	% se previene accesos no autorizaofs a aprtir de la anonimizacion de datos personales y la aplicacion de redes permisionadas
	\item Transparencia: El hecho de que las blockchains sean descentralizadas implica que cualquier miembro pueda visualizar en todo momento los datos que en ellas se registran. Esto permite comprobar su correctitud, disminuyendo las posibilidades de fraudes que suelen acaecer en otros escenarios.
	
	\item Trazabilidad: Este punto est\'a muy relacionado al anterior. Precisamente por la transparencia de la red es que cada usuario es capaz de verificar el historial de todas las transacciones ejecutadas sobre esta. N\'otese que, debido a que dichas redes son seguras e immutables, se garantiza la existencia de un registro con los cambios efectuados sobre ellas. As\'i, es posible conocer cada estado por el que ha transitado un activo y realizar auditor\'ias para comprobar que todo marche acorde a lo deseado.	

%aqui alcaraciond e Kmilo sobre como las blockchains son immutables, y como para hacer alguna modificacion, al ser registros de just add, se registra el cambio con una referencia al activo a modificar
\end{enumerate}

%desventajas
%INFO DEL TIMELINE 
%immutability, privacy, security, and transparency

\section{Nociones B\'asicas de Hyperledger Fabric}

Hyperledger Fabric (HLF) es un proyecto de c\'odigo libre del consorcio Hyperledger, creado a su vez por la \emph{Linux Foundation} [\cite{linuxfoundation}] con el objetivo de impulsar el desarrollo de tecnolog\'ias blockchain. Consiste en una DLT dise\~nada espec\'ificamente para entornos empresariales, atendiendo a las necesidades de identificar a los participantes en la red, establecer una serie de permisos para la interacci\'on, contar con privacidad y confidencialidad de transacciones, baja latencia en la confirmaci\'on de estas y capacidad para enviar una cantidad elevada de datos por unidad de tiempo (\emph{throughput}).

HLF surge entonces como plataforma privada y permisionada. A diferencia de los sistemas abiertos en los que cualquier desconocido puede consultar y modificar los datos, Fabric utiliza un Proveedor de Servicios de Membres\'ia (MSP) para poder identificar a cada usuario que se una a la red. As\'i, limita la interacci\'on para s\'olo un n\'umero de elementos autorizados y establece facilidades para otorgar permisos espec\'ificos a cada uno. Ofrece adem\'as una caracter\'istica muy novedosa y \'util, d\'igase la utilizaci\'on de canales como especie de subredes para compartir informaci\'on confidencial entre pares o grupos de miembros que lo requieran.

Fabric posee una arquitectura altamente modular y configurable, capaz de ser adaptada a los diversos requerimientos que puedan surgir en la industria. Por ejemplo, facilita customizar el protocolo de consenso a utilizar para determinar el orden de las transacciones. Con este fin provee implementaciones de mecanismos crash fault-tolerant (CFT) como Raft\footnote{mecanismo CFT actual y recomendado por HLF}, Kafka y Solo\footnote{mecanismo CFT para versiones de prueba}, lo cual permite a los usuarios elegir el m\'as adecuado al nivel de confianza existente entre ellos.
%Aqui pueden referenciarse varios articulos sobre los mecanismo de ordenaicon, solo que considero no son tan relevantes al tema tratado

De igual forma cuenta con protocolos enchufables de gesti\'on de identidad como LDAP u OpenID Connect, con protocolos de gesti\'on de claves o bibliotecas criptogr\'aficas, entre otros.

\nomenclature[ldap]{\textbf{LDAP}}{LDAP es un servicio de directorio que permite que los sistemas compartan información de usuarios y grupos. LDAP se puede utilizar para autenticar usuarios y administrar cuentas y grupos de usuarios.}

\nomenclature[oidc]{\textbf{OpenID Connect}}{Protocolo para autenticar usuarios y emitir tokens. Permite que dos partes, como un sitio web y un usuario, intercambien información de forma segura. Cuando un usuario inicia sesión en un sitio web, OpenID Connect puede verificar que el usuario es quien dice ser y emitir un token que se puede usar para acceder a los recursos protegidos en el sitio web.}

Adem\'as cabe destacar que, a diferencia de otras blockchains, HLF no requiere del uso de criptomonedas\footnote{aunque podr\'ian ser implementadas en el caso deseado} para incentivar el minado. Lo anterior podr\'ia considerarse como una medida de seguridad adicional, pues evita atraer atacantes que quieran socavar la estabilidad de la red. Adem\'as implica un costo operacional menos elevado de la plataforma.

\subsection{Componentes principales}

\subsubsection{Ledger}
El ledger o libro mayor de cuentas de Fabric est\'a conformado por dos elementos: Un \emph{world state} o estado mundial y un \emph{transaction log} o lista de transacciones. El primero resulta ser una suerte de base de datos del ledger, pues describe el estado de este en un instante de tiempo dado. Por su parte, el segundo, mantiene un registro del historial de transacciones que han resultado en el estado actual del world state.

Cada participante conserva una copia del ledger por cada red de HLF  a la que pertenece.

\subsubsection{Assets}
Mientras en blockchains como Bitcoin se mantiene un registro de las transferencias de criptomonedas, en Fabric la definici\'on de \textquestiondown qu\'e es transferido? resulta un poco m\'as flexible. En el \'ultimo caso se definen activos o \emph{assets} a transferir, lo que da cabida a un n\'umero de usos m\'as extendido.
%aqui se puede argumentar

Los activos pueden variar desde elementos tangibles como bienes ra\'ices y hardware, hasta intangibles como contratos y propiedades intelectuales. Hyperledger Fabric brinda la capacidad de modificar activos mediante transacciones de c\'odigo de cadena.

Los activos representan precisamente la colecci\'on de pares clave-valor, cuyos cambios de estado se registran como transacciones en el libro mayor.

\subsubsection{Chaincode}

Los \emph{smart contracts} son llamados \emph{chaincode} o c\'odigo de cadena en Fabric. Estos constituyen la l\'ogica de negocio de una aplicaci\'on blockchain. 

Los chaincodes definen activos y transacciones para modificarlos. Pueden ser desplegados din\'amicamente y correr de forma concurrente en la red.

Los elementos participantes de cada red permisionada en HLF son capaces de utilizarlos para interactuar con el ledger y transformarlo.

\subsubsection{Nodos}

Los nodos constituyen las entidades de comunicaci\'on de una blockchain. En HLF existen 3 tipos:

\begin{enumerate}
	\item Nodos Clientes: Los clientes que env\'ian invocaciones de transacciones a los \emph{endorsers} y transmiten las propuestas de transacciones al servicio de ordenaci\'on.
	
	
	\item Nodos Pares: O \emph{peers}, qui\'enes ejecutan transacciones y mantienen el estado y una copia del ledger. Pueden asumir el rol de endosadores o endorsers.
	
	\item Nodos Ordenadores: El conjunto de estos nodos conforman lo que se le llama servicio de ordenamiento, el cual se encarga de ordernar las transacciones endosadas y empaquetarlas en bloques. De igual forma se ocupan de darle un orden a dichos bloques siguiendo un protocolo de consenso como se comentaba anteriormente. Su segundo objetivo consiste en distribuir dichos bloques entre los peers para su posterior validaci\'on.
	
\end{enumerate}

Es importante destacar c\'omo el hecho de separar el endosamiento de la ejecuci\'on de chaincode (en los nodos peers) del servicio de ordenamiento, evita posibles cuellos de botella en el procesamiento de las transacciones, lo cual le provee a HLF ventajas de desempe\~no y escalabilidad sobre los sistemas que ejecutan ambas funciones en los mismos nodos.
%el embotellamiento
%arquitectura de HLF?
 
\subsection{Flujo}

La mayor\'ia de las blockchains utilizan un paradigma denominado como \emph{order-execute}. Con el fin de alcanzar el consenso, dicho paradigma requiere que los smart contracts a utilizar sean deterministas. Lo anterior limita la programaci\'on de smart contracts a lenguajes de dominio espec\'ifico que eliminen las operaciones no deterministas. 

La implementaci\'on de un flujo order-execute implica adem\'as que todas las transacciones se ejecuten secuencialmente por todos los nodos. Luego, ante posibles ataques maliciosos que deseen desequilibrar la red, han de ser tomadas medidas de protecci\'on extra.
%yo creo que aqui de lo que se habla es de q si hay un error en el smart contract ejecutado se replica en todos los nodos de la red

En cambio, en Fabric se utiliza un arquitectura conocida como \emph{execute-order-validate}, que resuelve muchos de los problemas comentados.

Dicho modelo consta de tres pasos:
\begin{enumerate}
	\item Primeramente, se ejecuta una transacci\'on, se chequea su correctitud y se respalda (proceso de \emph{endorsement}).
	
	\item Luego se ordenan las transacciones a partir del protocolo de consenso.
	
	\item Se validan teniendo en cuenta la pol\'itica de endosamiento espec\'ifica a la aplicaci\'on manejada y, por \'ultimo, se aplican al ledger.
\end{enumerate}

Las pol\'iticas de endosamiento a las que se hace referencia dictan qu\'e nodos par y cu\'antos de ellos deben garantizar la correcta ejecuci\'on de un contrato inteligente determinado. De esta forma se asegura que solo un subconjunto de los peers deba ejecutarlo. Esto facilita que el sistema sea m\'as escalable y tenga un mejor desempe\~no con la capacidad a\~nadida de ejecuciones en paralelo.

Igualmente, ha de notarse c\'omo se elimina el no determinismo, al poder filtrar los resultados inconsistentes antes de pasar a la fase de ordering.

Dicho cambio de arquitectura constituye una de las caracter\'isticas que realzan las ventajas de HLF ante otras blockchains. Este nuevo modelo que implementa, hace posible la programaci\'on de chaincode en lenguajes de propósito general como Python, Java y Javascript. V\'ease las facilidades que le brinda al programador para crear un v\'inculo con esta nueva DLT, ahorr\'andole horas de estudio para aprender un nuevo lenguaje de programaci\'on y de depuraci\'on para identificar errores. %nunca tratados por \'el.

%est\'andar por de propostio general

Precisamente en dicha capacidad de Fabric se fundamenta el presente proyecto, la cual har\'a posible la programaci\'on de smart contracts en C\# tras la implementaci\'on del engranaje apropiado.
 
\section{Mecanismos de autenticaci\'on y\\ autorizaci\'on en Fabric}
%considrar un nombre mas  apropiado (Servicios de membresia)
Como se ha mencionado anteriormente, una de las caracter\'isticas m\'as destacadas de HLF resulta el hecho de ser una red privada y permisionada. Esto significa que opera sobre un conjunto de participantes verificados que solo pueden interactuar con la red cuando el administrador oportuno le ha concedido los permisos requeridos. 

Bajo este principio de governanza se garantiza cierto nivel de confianza, puesto que, aunque los usuarios no se conozcan entre s\'i, toda acci\'on realizada queda registrada identificando de forma un\'ivoca a su ejecutor. As\'i, en caso de detectarse alguna anomal\'ia, es posible identificar y rastrear al responsable, y tomar las medidas necesarias para solucionar el incidente.

A continuaci\'on, se describen con un poco m\'as de profundidad los mecanismos implementados por Fabric para asegurar el correcto funcionamiento de este modelo permisionado.

\subsection{Infraestructura de Clave P\'ublica (PKI)}
Una infraestructura de clave p\'ublica (PKI del ingl\'es Public Key Infraestructure) es una colección de tecnolog\'ias de Internet que proporciona comunicaciones seguras en una red. Enlaza llaves criptogr\'aficas con entidades y provee los servicios necesarios para la correcta administraci\'on de certificados digitales y el cifrado de clave p\'ublica.

Cada PKI cuenta elementos claves como son:
\begin{enumerate}
	\item Autoridad certificadora (CA): Encargada de emitir certificados de clave p\'ublica para una entidad a modo de confirmaci\'on de sus credenciales.
	
	\item Llaves p\'ublicas y privadas: Indispensables para el funcionamiento de los mecanismos de firma digital.
	
	La llave privada que ha de mantenerse secreta, es la utilizada para producir firmas digitales sobre mensajes. La p\'ublica se deja disponible a qui\'en desee consultarla y es la que funciona como ancla de autenticaci\'on. Se establece una relaci\'on matem\'atica entre ambas llaves de forma tal que, la firma producida para un mensaje con una llave privada espec\'ifica solo es v\'alida bajo su correspondiente llave p\'ublica y para el mensaje dado. As\'i, es posible verificar la integridad de la informaci\'on transmitida y la autenticidad de su remitente.
	
	\item Certificados digitales: Documentos que contienen un conjunto de propiedades relacionados con su titular. Incluyen la llave p\'ublica de la parte involucrada y, opcionalmente, atributos del titular de la llave privada (d\'igase roles u otros), fecha de vigencia, y la firma digital de la CA emisora.
	%el tipo de certificado mas comun es el x.509 que permite a la parte que codifica especificar detalles en su estructura
	
	\item Lista de revocaci\'on de certificados: Del ingl\'es Certificate Revocation List (CRL), constituye una lista generada por la CA que referencia a los certificados que ya no son v\'alidos. Pueden ingresar a la lista certificados que hayan sido comprometidos o aquellos cuyo material criptogr\'afico se haya visto expuesto.
	
\end{enumerate}

En Fabric se adopta un modelo jer\'arquico de infraestructura de clave pública tradicional, el cual, junto a una implementaci\'on de MSP permite el manejo correcto de identidades. En los siguientes ep\'igrafes se analiza mejor este engranaje.

\subsection{Autoridades Certificadoras}
Una autoridad certificadora (CA del ingl\'es certificate authority) no es m\'as que una entidad capaz de producir, firmar y almacenar certificados digitales.

Por lo general el flujo que se sigue para el intercambio pasa por:

\begin{enumerate}
	\item Una parte emite una solicitud de certificado digital a partir de la generaci\'on de su llave privada (que no devela ni a la CA), su llave p\'ublica y una solicitud de firma de certificado (CSR del ingl\'es certificate signing request) con la informaci\'on que se requiere plasmar en \'el.
	
	\item La CA chequea que los datos recibidos sean correctos y emite un certificado que firma con su clave privada.
	
	\item Finalmente, terceros pueden verificar la autenticidad de la entidad inicial a partir de la llave p\'ublica de la CA involucrada.
\end{enumerate}


\subsection{Fabric CA}
%Because Fabric CA is a custom CA targeting the Root CA needs of Fabric, it is inherently not capable of providing SSL certificates for general/automatic use in browsers.

Aunque HLF puede utilizar cualquier autoridad certificadora facultada para generar certificados X.509, cuenta con una implementaci\'on propia llamada Fabric CA. Su uso es opcional, no obstante, se recomienda elegir esta alternativa por estar capacitada para la creaci\'on de la estructura de carpetas MSP local y la organizaci\'on requerida por HLF.%certificados ECDSA ademas de haber sido testeada y comprobada su eficacia en este sentido

Provee funcionalidades tales como:
\begin{enumerate}
	\item Registro de entidades o capacidad de conexi\'on a un protocolo ligero de acceso a directorios (LDAP o \emph{Lightweight Directory Access Protocol}) que funcione como registro de usuarios.
	
	\item Emisi\'on de certificados de enrolamiento o incripci\'on para firmas e identificaciones.%ECerts enrollment certf
	
	\item Expedici\'on de certificados de transacci\'on.
	
	\item Anonimato y desvinculaci\'on al realizar transacciones.
	
	\item Renovaci\'on y revocaci\'on de certificados.
\end{enumerate}

La Fabric CA est\'a conformada por dos componentes: un servidor y un cliente. En la imagen \ref{fig:cadiagram} se puede apreciar c\'omo el servidor Fabric CA encaja en la arquitectura general de Hyperledger Fabric.

\begin{figure}[h]
	\centering
	\includegraphics[scale=0.3]{Graphics/fabric-ca}
	\caption{Fabric CA en comunicaci\'on con una red de Hyperledger Fabric.}
	\label{fig:cadiagram}
\end{figure}

Para interactuar con la Fabric CA se pueden utilizar o bien el Fabric CA client [\cite{caclient}] o el SDK correspondiente al lenguaje con el que se desarrolle la aplicaci\'on en concreto. Todas las comunicaciones se realizan a trav\'es de REST APIs [\cite{restapifaclient}], s\'olo que los componentes anteriores funcionan como interfaces al servidor. Resultan un intermediario m\'as adecuado ante los no tan amigables pedidos REST.

El cliente o el fabric sdk pueden conectarse a un servidor en un grupo de servidores de HLF. En la imagen \ref{fig:cadiagram} se puede observar c\'omo el cliente accede a un \emph{endpoint} (HA Proxy) que equilibra la carga del tr\'afico hacia uno de los miembros del cl\'uster. Todos los miembros comparten una misma base de datos para el almacenamiento de identidades y certificados. En caso de configurarse el LDAP, la informaci\'on referente a las identidades se mantiene en un registro de este tipo.

Por defecto, el servidor de Fabric CA cuenta con una sola CA. No obstante, es posible configurarlo para que haga uso de m\'ultiples CAs, encaden\'andolas entre s\'i para establecer una red de mayor confianza. As\'i, los certificados ra\'ices pueden ser expedidos por una CA padre u otra intermediaria, lo que permite que la red escale m\'as f\'acilmente y protege a su CA raiz.%incluso es posible deshabilitar la root ca una vez las intermediarias esten en funcionamiento, lo que reduce su vulnerabilidad

Para cada red de HLF se recomienda el despliegue de dos autoridades certificadoras por organizaci\'on, una CA de organizaci\'on y una CA de TLS (\emph{Transport Layer Security}). Aunque estas no difieren en funcionamiento, s\'i lo hacen el tipo de certificado que expiden. La primera es la encargada de generar identidades para nodos y organizaciones, mientras que la segunda es la que asegura las comunicaciones en la red.

Por lo general se aconseja tambi\'en separar al servicio de ordenamiento en una organizaci\'on diferente a las de los nodos pares con su propia CA. En caso de existir nodos ordenadores agrupados por organizaciones, ha de garantizarse una CA propia para cada una.

Toda esta separaci\'on de funcionalidades permite distribuir las gestiones lo m\'as posible, lo cual reduce las posibilidades que tendr\'ia un atacante ante la red y resulta indispensable en general para minimizar las vulnerabilidades de la blockchain.

%garantizando que el material criptogr\'afico que proveen estas autoridades sea de total confianza.

\subsection{MSP}
Mientras las autoridades certificadoras generan identidades a partir de un par de llaves p\'ublicas y privadas, es necesario un mecanismo para reconocer dichas entidades en la red. Aqu\'i es donde aparece en escena el concepto de Provedoores de Servicios de Membres\'ia o MSP del ingl\'es \emph{Membership Service Providers}.

Los MSP son componentes de HLF que definen las reglas que rigen a las entidades v\'alidas para una organizaci\'on espec\'ifica. Permiten convertir identidades en roles; son qui\'enes determinan qu\'e permisos tiene cada entidad a nivel de organizaci\'on, nodo y canal; y definen las organizaciones (inclu\'idas CAs) en las que conf\'ian los miembros de una red. Por ello, se consideran encargados de validar las acciones realizadas por cada entidad, velando siempre porque cada una realice solo aquellas a las que se le ha dado acceso.

A diferencia de lo que su nombre puediera indicar, un MSP no provee nada. Su implementaci\'on concreta consiste en una serie de carpetas que se a\~naden a la configuraci\'on de la red y permite definir a cada organizaci\'on tando a nivel interno (declarando por ejemplo a sus administradores) como externo (permitiendo que otras organizaciones validen que las entidades posean la autoridad para realizar las acciones que solicitan).

Existen dos tipos de configuraciones MSP, una a nivel local y otra de canal. La divisi\'on entre estas refleja las necesidades de las organizaciones para administrar sus recursos locales, como un par o nodos de pedido, y sus recursos de canal, como contratos inteligentes y otros que operen a nivel de canal o red.

Una carpeta MSP local est\'a compuesta por:

\begin{enumerate}
	\item Un archivo \emph{config.yaml} utilizado para configurar la clasificaci\'on de identidades en Fabric al habilitar el \emph{Node OUs} y definir los roles aceptables.
	
	\nomenclature[nodeou]{\textbf{Node OUs}}{Las unidades organizativas de nodo son una forma de agrupar nodos en un HLF. Se utilizan para administrar el ciclo de vida de los nodos, incluido el inicio, la detención y la administración del estado de los nodos. Las identidades pueden usar estos NodeUS para clasificarse como cliente, administrador, par o ordenante. Las cuatro clasificaciones son mutuamente excluyentes.}
	
	\item Una carpeta \emph{cacerts} con una lista de certificados autofirmados X.509 de las CAs ra\'iz en las que conf\'ia la organizaci\'on a la que representa el MSP.
	
	\item Una carpeta \emph{intermediatecerts} con certificados X.509 de las CAs intermedias en las que conf\'ia la organizaci\'on. Cada certificado debe estar firmado por un CA de la carpeta de ra\'ices confiables o bien otro intermediario cuya cadena termine en el mismo sitio. Ambas carpetas definen las CAs de las que deben provenir los certificados para considerarse miembros de la organizaci\'on, aunque a diferencia de la primera, la de intermediarios puede estar vac\'ia.
	
	\item Una carpeta de \emph{admincerts} que lista las identidades que definen actores con el rol de administradores en la organizaci\'on. En versiones posteriores a la 1.4.3 no existe; en cambio, se definen los administradores en el momento de creaci\'on de la entidad, cuando la CA determina el valor de la propiedad \emph{Node OU rol} del certificado.
	
	
	\item Una carpeta \emph{keystore} con la llave privada del nodo correspondiente (ya sea par, ordenador o cliente). Dada su utilidad para la firma de transacciones y respuestas se limita su acceso a los administradores del nodo en s\'i. 
	
	\item Para nodos pares u ordenadores una \emph{signcert} que contine el certificado generado por la CA. Representa su identidad y junto a su llave privada correspondiente puede ser utilizada para la creaci\'on de firmas.
	
	\item Un par de carpetas \emph{tlscacerts} y \emph{tlsintermediatecerts} \'idem a las 2 y 3 pero con los certificados de las CAs para asegurar las comunicaciones con TLS.
	
	\item Por \'ultimo, una carpeta \emph{operationcerts} con los certificados requeridos para establecer comunicaci\'on con la API de \emph{Fabric Operations Service} (destinada a los operadores de la red).
\end{enumerate}

Puesto que en el caso del MSP de canal no se necesitan habilidades de firma, sino de verificaci\'on de entidades, se excluyen los directorios de keystore y signcert. Adem\'as se a\~nade una carpeta extra, \emph{Revokes Certificates}, con informaci\'on referente a las entidades de actores que hayan sido revocadas. Para certificados del tipo X.509 se guardan pares de \emph{strings} conocidos como \emph{Subject Key Identifier} (SKI) y \emph{Authority Access Identifier} (AAI). Dicha lista es conceptualmente id\'entica a la CRL de una CA, pero se relaciona tambi\'en con la revocaci\'on de membres\'ia de la organizaci\'on.


\section{Fabric SDKs y Fabric Gateways} 

Las aplicaciones que deseen interactuar con el ledger han de hacerlo por medio de un intermediario, un \emph{Software Development Kit (SDK)} o paquete de desarrollo de \emph{software}. Los espec\'ificos a HLF est\'an dise\~nados para comunicarse con los smart contracts, encargados a su vez de realizar las operaciones de \emph{get, put, delete} sobre el world state y registrar las transacciones en la blockchain.%deben hacerlo

Hyperledger Fabric ofrece un n\'umero de SDKs para el desarrollo de aplicaciones en una amplia variedad de lenguajes de programaci\'on. Actualmente cuenta con soporte para Node.js [\cite{sdknode}] y Java [\cite{sdkjava}], los primeros en aparecer. Igualmente, a pesar de no estar tan desarrolladas como las primeras, ofrece versiones para Go [\cite{sdkgo}] y Python [\cite{sdkpython}].

Vale aclarar que estos paquetes son considerados SDKs de bajo nivel. Contienen una considerable cantidad de c\'odigo y, por tanto, una complejidad bastante elevada. Queda en manos del programador con la intenci\'on de conectar una aplicaci\'on con Fabric, comprender el funcionamiento de las solicitudes de transacciones y el endosamiento para implementarlos a partir de estas plataformas, dado que en realidad lo que contienen no es m\'as que una adaptaci\'on de las llamadas gRPC [\cite{grpc}] para cada lenguaje espec\'ifico.

Para resolver estos problemas se han creado APIs de alto nivel que le permiten a los desarrolladores abstraerse un poco m\'as del funcionamiento de HLF en sus or\'igenes. As\'i, con proyectos como los \emph{Fabric Gateways} [\cite{fabricgateway}] y [\cite{fabricgatewayjava}] se facilita la realizaci\'on de consultas, ejecuci\'on de transacciones y respuesta a eventos que permitan conocer y modificar el estado del ledger.

El surgimiento de estos \'ultimos modelos m\'as consistentes, supresores de mucho c\'odigo repetitivo propenso a errores, ha sido bien recibido por la comunidad. No obstante, su funcionamiento cuenta en las bases con los SDKs mencionados al inicio, siendo de hecho construidos encima de estos. Adem\'as carecen de un mecanismo para gestionar tareas de administraci\'on. Por ello, a\'un se hace necesario intercambiar directamente con los SDKs de bajo nivel para la creaci\'on de identidades, su registro entre otras operaciones de relevancia.

Dado que el objetivo del presente trabajo gira en torno a estas funcionales, son entonces los SDKs originales de Java, Node, Go y Python los que fungen como gu\'ia para el desarrollo.

\subsection{Caracter\'isticas de los Fabric SDKs}

En general los 4 repos de Fabric SDK existentes siguen las especificaciones de [\cite{fabricsdkspec}], brindando las primitivas para acceder a las funcionalidades b\'asicas de la red desde una capa de abstracci\'on que aligera la carga del usuario. Permiten a los desarrolladores interactuar con la blockchain para desplegar e invocar chaincode, escuchar eventos que se puedan generar y recuperar informaci\'on acerca de los bloques y transacciones salvaguardados en el ledger. 

%following paragraph should be supported with examples
El SDK de Go resulta un poco diferente al resto en cuanto a estilo e implementaci\'on; posee un mayor nivel de abstracci\'on que le facilita al programador el proceso de acoplamiento con la plataforma. Los de Java y Node disponen de una capa extra que separa las funciones para el desarrollo de aplicaciones de las administrativas. As\'i, aunque cada uno le aporta las caracter\'isticas propias del lenguaje en particular y le adiciona las modificaciones consideradas prudentes, todos se ajustan a la misma idea central.

% diesnnado en un estilo orientado a objeto. su disenno modular permite a los desarrolladores unir sus implementaciones alternativas para manehar eventos de ocnfirmacion de trasanacciones , evaluacion o query,e tc

Tomemos como ejemplo el de Node.js para un an\'alisis a mayor profundidad. Este est\'a dise\~nado siguiendo el estilo de la programaci\'on orientada a objetos. Se compone de 3 m\'odulos:

\begin{enumerate}
	\item \texttt{fabric-network}: El cual provee APIs de alto nivel para que las aplicaciones clientes interact\'uen con el chaincode, siendo el recomendado para la contrucci\'on de aplicaciones.
	
	\item \texttt{fabric-ca-client}: Que permite la comunicaci\'on con la autoridad certificadora de Fabric.
	
	\item \texttt{fabric-common}: Una API de bajo nivel utilizada para implementar la capacidad de red de fabric-network. Facilita la interacci\'on con los componentes claves de una red de HLF, d\'igase nodos pares, ordenadores y flujos de eventos.
\end{enumerate}

%[esto puede ponerse aqui o pasarlo a la parte de propuesta]
%Escribir aqui un poquito de las notas que hay en la libreta sobre que hace cada repo

El m\'odulo \texttt{fabric-ca-client}, el de mayor inter\'es para este proyecto, es el que provee las capacidades para interactuar con la Fabric CA para el manejo de entidades.  Permite registrar usuarios, inscribirlos para obtener el certificado correspondiente firmado por la CA,  revocar certificados, revocar usuarios seg\'un su \emph{enrollmentId} o n\'umero identificador de inscripci\'on, entre otros.

Para poder acceder a estas funcionalidades la biblioteca cuenta con la clase 
\\
\texttt{ FabricCAServices} que debe ser instanciada como punto de entrada. Esta, poseedora a su vez de una clase \texttt{FabricCAClient}, expone los m\'etodos para gesti\'on de certificados y entidades plasmados en la segunda. Posee adem\'as una implementaci\'on de la interfaz \texttt{CryptoSuite}, requerida por cada SDK, que encapsula los algoritmos para firmas digitales, encriptaci\'on y hashing seguro. En este caso el repo de Node.js ofrece implementaciones para ECDSA y SHA2/3, aunque admite que se utilicen otras alternativas si son especificadas en las configuraciones de la CrytoSuite.

%Comentar que JAVA es igual orientado a objetos lo que no cuenta con la clase FCAServices

%Puede hablarse de X509 certificate

%Ver en Member Service Provider interface and the COP implementation que definen el enroll y el resto

%Otros detalles ineresantes como concurrencia...
\subsection{Fabric SDK para C\#}

Hasta el momento el consorcio de Hyperledger no ha revelado trabajos en curso para extender la lista de lenguajes que soporta HLF. Sin embargo, resulta de interés poder disponer de la capacidad de programación de contratos inteligentes en otros lenguajes de amplio uso, como por ejemplo C\#.

Qu\'e elementos concretos de C\# lo hacen ideal para ingresar a esta lista?

C\# es un lenguaje de de programación de propósito general, con un modelo orientado a objetos bien definido. Entre la serie de ventajas que ofrece este potente lenguaje se incluyen:

\begin{enumerate}
	\item Facilidad de aprender y usar, por lo que resulta una excelente opción para principiantes.
	%curva de aprendizaje suave
	\item Sus fuertes lazos con la familia de lenguajes C, lo que hace menos complejo que cualquier ingeniero con una sólida comprensión de C y C++ pueda hacer el cambio.
	
	\item Su versatilidad, se suele utilizar para la creaci\'on una amplia gama de aplicaciones.
	
	\item El ser un lenguaje maduro, lo que significa que cuenta con un buen soporte y una gran comunidad de desarrolladores. El hecho de que sea un producto original de Microsoft implica que cuenta con el respaldo del gigante tecnológico, lo que se traduce en ayuda de expertos, recursos adicionales y actualizaciones frecuentes. Adem\'as con el tiempo ha reunido un amplio conjunto de bibliotecas y herramientas que facilitan la creación de aplicaciones sólidas, menos propensas a errores.
\end{enumerate}

Dichas razones, en conjunto con la no existencia de trabajos hasta el momento que acaten esta tarea, el objetivo general de este trabajo de tesis se centra en la creaci\'on de un Fabric SDK para C\#. En el pr\'oximo cap\'itulo se presenta la propuesta de un m\'odulo \emph{FabricCaClient} de dicho SDK para entablar comunicaci\'on con una CA de Fabric. En el futuro se espera dicho proyecto sea extendido para, junto a [\cite{chaincode22csharp}] permitan an\~adir soporte completo para la programaci\'on sobre Hyperledger Fabric en C\#.
 
\chapter{Propuesta}\label{chapter:proposal}
Desarrollo
•Fundamentación teórica (Marco 
teórico)
\chapter{Experimentos y Resultados}\label{chapter:implementation}
% nativo 
Hasta aqu\'i se han expuesto los elementos fundamentales en la creaci\'on de un kit de desarrollo de software para C\# que cubra las funcionalidades b\'asicas para la comunicaci\'on con una CA de HLF. A continuaci\'on se presentan algunos de los experimentos llevados a cabo para probar su eficacia, a la vez que se indica la forma para interactuar con esta plataforma. Finalmente se comentan los resultados generales de este proyecto.

Para testear el funcionamiento del SDK implementado se llevaron a cabo numerosos experimentos contra un servidor de Fabric CA. Los m\'as relevantes fueron registrados en un proyecto separado dentro de la soluci\'on, nombrado \texttt{Test}. En este proyecto se recogieron una serie de pruebas unitarias (del ingl\'es \emph{Unit Tests} [\cite{unittestvs}]) sobre componentes aislados, desglosando la funcionalidad de cada clase en comportamientos discretos que fuera posible probar como unidades individuales.

En la soluci\'on propuesta se agregan dos archivos de tests: \texttt{CAServiceTests}, que re\'une las pruebas relacionadas con la clase \texttt{CaService} y \texttt{CAClient}; y un \texttt{WalletTests}, que analiza todo lo referente a las billeteras. Aunque los nombres de los archivos de prueba pueden ser arbitrarios, en este caso se adopta una convención de nomenclatura que utiliza el nombre de la clase a probar con la terminaci\'on Test. Lo mismo para cada m\'etodo espec\'ifico, a\~nadiendo tambi\'en en los casos necesarios una mini descripci\'on del objetivo o el comportamiento esperado. Se espera este estándar facilite la comprensi\'on del c\'odigo para futuros desarrolladores del proyecto, y que sea utilizado para los nuevos aportes.
% de la prueba especifica.

En la clase \texttt{CAServiceTests} se implementaron tests para chequear comportamientos como: 

\begin{enumerate}
	\item Que a trav\'es una llamada de registro fuera posible inscribir a un usuario.
	
	\item La negaci\'on del registro a una identidad tras haberle revocado sus permisos.
	
	\item Que al proveer un password o secret este fuera utilizado para el registro ante la CA [\ref{code:regsecret}].
	
	\begin{lstlisting}[caption={Test de la clase \texttt{CAServiceTests} para comprobar que el secret provisto se utilice en el registro.}, label={code:regsecret}]
	[TestMethod()]
	public async Task RegisterWithSecretProvidedTest() {
	string userId = "appUser";
	string usrpw = userId + "pw";
	int maxEnrollment = 5;
	
	string secret = await caService.Register(userId, usrpw, maxEnrollment, null, adminEnr);
	
	Assert.AreEqual(secret, usrpw, "Registration call didn't set provided secret.");
	}
	\end{lstlisting}
	
	\item La posibilidad de reinscribir a un usuario con los datos previos.
	
	\item Que al proveer un csr el registro sea efectuado correctamente.
	%que se registren los atributos provistos etc. 
\end{enumerate}

Por otro lado, la clase \texttt{WalletTests} se centra en comprobar que las operaciones caracterist\'icas de una billetera realmente modifiquen el sistema de archivos. Por ejemplo, el test \ref{code:walletput} intenta a\~nadir al wallet una identidad en representaci\'on de un admin ya inscrito, comprobando luego que el directorio contenga la informaci\'on al respecto.

%aclarar que antes ya se habi creado el wallet
\begin{lstlisting}[caption={Test para chequear el m\'etodo \texttt{Put} de la clase \texttt{Wallet}.}, label={code:walletput}]
[TestMethod()]
public void PutTest() {
	FSWalletStore.ClearDirectory(storagePath);

	var idenList = wallet.List();
	Assert.AreEqual(idenList.Length, 0, "Initially a wallet should contain no values.");

	// put element in wallet
	X509Identity identity = new X509Identity(adminEnr.Cert, adminEnr.PrivateKey, orgMSP);
	wallet.Put(registrarName, identity);

	// check element was saved
	var newIdenList = wallet.List();
	Assert.AreEqual(newIdenList.Length, 1, "Wallet should contain just the element saved.");
	Assert.AreEqual(newIdenList[0], registrarName, "Wallet content doesn't match the expected file.");

	// clear wallet
	wallet.Remove(registrarName);
}
\end{lstlisting}

%[describir en el ejemplo El patrón AAA (Organizar, Actuar, Afirmar) es una forma común de escribir pruebas unitarias para un método bajo prueba.]

Importante tambi\'en fue el realizar pruebas de integraci\'on (\emph{Integration Tests}) para evaluar los componentes de la aplicación en un nivel más amplio. Estas se enfocan en verificar que los elementos dispersos del SDK funcionen en conjunto para producir el resultado esperado.

Por ejemplo, continando con el \texttt{Wallet}, resulta evidente la necesidad de comprobar que, tras cargar una identidad previamente guardada, sus valores originales no se vean corrompidos ni se afecte el flujo de ejecuci\'on. En el fragmento de c\'odigo \ref{code:walletFlow} se almacena en la billetera la identidad de un admin, se recupera y se realizan una serie de operaciones bajo su comando (d\'igase el registrar un nuevo usuario, inscribirlo y revocarle luego sus permisos).

\begin{lstlisting}[caption={Test para verificar el comporamiento de un flujo completo utilizando identidades de una billetera.}, label={code:walletFlow}]
[TestMethod()]
public async Task CaClientFlowWithWalletTest() {
	string userId = "appUser1";
	string userSecret = "xsw";
	int maxEnrollment = 5;

	FSWalletStore.ClearDirectory(storagePath);

	X509Identity identity = new X509Identity(adminEnr.Cert, adminEnr.PrivateKey, orgMSP);
	wallet.Put(registrarName, identity);

	// retrieve identity
	var adminIdentity = wallet.Get(registrarName);
	Enrollment newAdminEnr = new Enrollment(adminIdentity.GetPrivateKey(), adminIdentity.GetCertificate(), null, caService);

	// register user
	string secret = await caService.Register(userId, userSecret, maxEnrollment, null, newAdminEnr);

	// enroll user
	Enrollment usrEnr = await caService.Enroll(userId, secret);

	// revoke credentials
	var result = await caService.Revoke(userId, "", "", "unspecified", true, newAdminEnr);

	try {
		// check user is unable to enroll after its credentials are revoked
		Enrollment newUsrEnr = await caService.Enroll(userId, secret);
		wallet.Remove(registrarName);
	}
	catch (EnrollmentException exc) {
		StringAssert.Contains(exc.ToString(), unauthorizedMessage);
		wallet.Remove(registrarName);
		return;
	}

	Assert.Fail("Expected an enrollment denial.");
}
\end{lstlisting}

N\'otese que todos estos tests resultan relevantes no s\'olo en el momento presente, sino tambi\'en para el posterior mantenimiento y extensi\'on del SDK. Se espera brinden seguridad al desarrollador en cuanto a que los cambios efectuados no interfieran en los viejos mecanismos ya establecidos.


%Para ejecutar Test Explorer, la mayoría de los marcos requieren que agregue atributos específicos para identificar los métodos de prueba unitaria.


%Estas pruebas más amplias se utilizan para probar la infraestructura de la aplicación y el marco completo, que a menudo incluyen los siguientes componentes:

%Base de datos, sistema de archivos, Dispositivos de red, Tubería de solicitud-respuesta

Debe aclararse que para este sistema bajo prueba ("SUT" seg\'un la terminolog\'ia utilizada por la documentaci\'on oficial) no se escribieron tests de integración para cada permutación de datos posibles. Solo se consideraron un conjunto de pruebas enfocadas en las funcionalidades b\'asicas del sdk y las rutinas m\'as comunes. Esto suele ser lo recomendado y generalmente suficiente para probar adecuadamente un sistema. 

%pruebas de integración de lectura, escritura, actualización y eliminación 



\backmatter

\begin{conclusions}
 El presente trabajo proporciona los primeros acercamientos para el dise\~no e implementaci\'on de un SDK que facilite el intercambio entre una red de Hyperledger Fabric y una aplicaci\'on cualquiera en el lenguaje de programaci\'on C\#. 
 
 Las ventajas de HLF como blockchain empresarial son innegables, por lo que resulta de inter\'es extender la lista de lenguajes a los que provee soporte.
 
 El documento cuenta con una breve rese\~na de las caractr\'isticas generales de la blockchain HLF, resaltando las ventajas que podr\'ian ofrecerse al extender dicha lista. Igualmente se presentaron los mecanismos de autenticaci\'on y autorizaci\'on en esta plataforma, los cuales constituyen la base de las funcionalidades a implementar.

 Para comprender el funcionamiento a replicar, fueron analizadas las bibliotecas \texttt{fabric-sdk} ya existentes para Python, Golang, Node.js y Java. Se evaluaron las estructuras y patrones utilizados, descartando algunos y conservando los m\'as adecuados para el lenguaje en cuesti\'on.
 
 A partir de los estudios realizados, fue posible dise\~nar e implementar el m\'odulo \texttt{FabricCaClient} de un  \texttt{fabric-sdk-csharp}, para generar identidades criptogr\'aficas con C\#.
 
 Para garantizar el soporte para las peticiones b\'asicas de registro, inscripci\'on, renovaci\'on y revocaci\'on de certificados fue necesario tambi\'en implementar funcionalidades para la creaci\'on de llaves criptogr\'aficas y generaci\'on de firmas digitales. estos aspectos fueron recogidos en la interfaz \emph{CryptoSuite} y la clase \emph{CryptoPrimitives}. 
 
 Por \'ultimo, como forma de garantizar la persistencia de datos, se implement\'o un wallet o billetera que permita a las aplicaciones almacenar y acceder a información sobre identidades de usuarios.
 
 El c\'odigo escrito fue objeto de pruebas, verificando que los resultados obtenidos se adecuaran al comportamiento esperado. Como resultado se arriv\'o a un un código robusto, flexible, libre de errores, capaz de satisfacer las
 funcionalidades básicas para las que fue concebido.
 
\end{conclusions}

\begin{recomendations}
Como se ha comentado, la implementaci\'on propuesta no constituye m\'as que un primer acercamiento para a\~nadir soporte al desarrollo de aplicaciones sobre Hyperledger Fabric utilizando C\#. Teniendo esto es cuenta, es posible indicar un conjunto de l\'ineas a seguir en funci\'on de acercar el trabajo futuro hacia al objetivo propuesto:

\begin{enumerate}
	%lo del user
	%extender test para cubirs otros comportamientos que pueden haberlse saltado
	\item Refactorizar el c\'odigo para encapsular ciertos comportamientos en clases espec\'ificas, como los par\'ametros requeridos en cada intecambio con la CA. Esto facilitar\'ia el control de errores y ajustar\'ia m\'as el al paradigma de orientaci\'on a objetos del lenguaje utilizado.
	
	\item Ampliar las funcionalidades de la clase \emph{CriptoPrimitives}, mediante la inclusi\'on de los m\'etodos de \emph{Encrypt}, \emph{Decrypt} y \emph{Verify}.
	
	\item Extender la clase \emph{WalletStore} e implementar la clase \emph{HsmX509Identity} para brindar a las apliciones un medio m\'as seguro donde salvaguardar su material criptogr\'afico.
\end{enumerate}

Las funcionalidades recomendadas responden fudamentalmente al trabajo efectuado hasta el momento. En el futuro se podr\'ia proseguir con otros elementos del m\'odulo \texttt{FabricNetwork} y \texttt{FabricCommon} que permitan conectarse a una red de Fabric, acceder a una canal e instanciar y consultar chaincodes.

\end{recomendations}

\include{BackMatter/Bibliography}
\include{BackMatter/Glossary}

\end{document}