\documentclass[12pt,oneside]{uhthesis}
\usepackage{subfigure}
%\usepackage[ruled,lined,linesnumbered,titlenumbered,algochapter,spanish,onelanguage]{algorithm2e}
\usepackage{amsmath}
\usepackage{amssymb}
\usepackage{amsbsy}
\usepackage{caption,booktabs}
\captionsetup{ justification = centering }
%\usepackage{mathpazo}
\usepackage{float}
\setlength{\marginparwidth}{2cm}
\usepackage{todonotes}
\usepackage{listings}
\usepackage{xcolor}
\usepackage{multicol}
\usepackage{graphicx}
\usepackage{caption}
%\captionsetup[table]{position=bottom} 
\floatstyle{plaintop}
%\restylefloat{table}
\addbibresource{Bibliography.bib}
% \setlength{\parskip}{\baselineskip}%
\renewcommand{\tablename}{Tabla}
%\renewcommand{\listalgorithmcfname}{Índice de Algoritmos}
%\dontprintsemicolon
%\SetAlgoNoEnd

\definecolor{codegreen}{rgb}{0,0.6,0}
\definecolor{codegray}{rgb}{0.5,0.5,0.5}
\definecolor{codepurple}{rgb}{0.58,0,0.82}
\definecolor{backcolour}{rgb}{0.95,0.95,0.92}
\definecolor{redstrings}{rgb}{0.9,0,0}
\lstdefinestyle{mystyle}{
	language=[Sharp]C,
	backgroundcolor=\color{white},   
	commentstyle=\color{codegreen},
	keywordstyle=\color{blue},
	morekeywords={partial, var, value, get, set, T},
	numberstyle=\tiny\color{codegray},
	stringstyle=\color{redstrings},
	%    basicstyle=\ttfamily\footnotesize,
	breakatwhitespace=true,         
	breaklines=true,                 
	captionpos=b,                    
	keepspaces=true,                 
	numbers=left,                    
	numbersep=5pt,                  
	showspaces=false,                
	showstringspaces=false,
	showtabs=false,                  
	tabsize=4,
	%    
	frame=lines,
	escapeinside={(*@}{@*)},
	%commentstyle=\color{greencomments},
	basicstyle=\ttfamily,
}

\lstset{style=mystyle}

%\lstdefinestyle{mystyle}{
%    backgroundcolor=\color{backcolour},   
%    commentstyle=\color{codegreen},
%    keywordstyle=\color{purple},
%    numberstyle=\tiny\color{codegray},
%    stringstyle=\color{codepurple},
%    basicstyle=\ttfamily\footnotesize,
%    breakatwhitespace=false,         
%   breaklines=true,                 
%   captionpos=b,                    
%    keepspaces=true,                 
%   numbers=left,                    
%    numbersep=5pt,                  
%    showspaces=false,                
%   showstringspaces=false,
%    showtabs=false,                  
%   tabsize=4
%}

%\lstset{style=mystyle}

\title{Módulo fabric-ca-client \\ del SDK de Fabric \\ para gestionar identidades criptográficas \\ con C\#.}
\author{\\\vspace{0.25cm}Gabriela B. Mart\'inez Giraldo}
%\advisor{\\\vspace{0.25cm}Ing. Daniel Mena \\\vspace{0.25cm}Ing. Camilo  Denis González\\\vspace{0.2cm}Dr. Miguel Katrib Mora}

\advisor{\\\vspace{0.25cm}Camilo Denis González
	\\\vspace{0.25cm}Daniel Mena Frias
\\\vspace{0.25cm}Miguel Katrib Mora}
\degree{Licenciado en Ciencia de la Computación}
\faculty{Facultad de Matemática y Computación}

\date{Enero del 2023\\\vspace{0.25cm}\href{https://github.com/ic-matcom/fabric-sdk-csharp}{github.com/ic-matcom/fabric-sdk-csharp}}

\logo{Graphics/uhlogo}

\makenomenclature

\renewcommand{\vec}[1]{\boldsymbol{#1}}
\newcommand{\diff}[1]{\ensuremath{\mathrm{d}#1}}
\newcommand{\me}[1]{\mathrm{e}^{#1}}
\newcommand{\pf}{\mathfrak{p}}
\newcommand{\qf}{\mathfrak{q}}
%\newcommand{\kf}{\mathfrak{k}}
\newcommand{\kt}{\mathtt{k}}
\newcommand{\mf}{\mathfrak{m}}
\newcommand{\hf}{\mathfrak{h}}
\newcommand{\fac}{\mathrm{fac}}
\newcommand{\maxx}[1]{\max\left\{ #1 \right\} }
\newcommand{\minn}[1]{\min\left\{ #1 \right\} }
\newcommand{\lldpcf}{1.25}
\newcommand{\nnorm}[1]{\left\lvert #1 \right\rvert }
\renewcommand{\lstlistingname}{Ejemplo de código}
\renewcommand{\lstlistlistingname}{Ejemplos de código}

\begin{document}

\frontmatter

\maketitle
%\begin{center}
	%\includegraphics[width=2.5cm]
	%{Graphics/qr.png}
	%\date{Fecha}
	%\\\vspace{0.4cm}
%	20 de enero de 2023
%\end{center}

\begin{dedication}
    Dedicación
\end{dedication}
\begin{acknowledgements}
	A mi mam\'a, la raz\'on por la que me levanto cada d\'ia. Mi hero\'ina.
	
	A mi pap\'a que ha sido siempre el mejor ejemplo de fortaleza y energ\'ia de vida.
	
	A este famili\'on inmenso que ha estado siempre de cerquita velando por m\'i, ese que no se deja vencer por muchos obst\'aculos que la vida disponga en el camino. A mis t\'ias Ara, Deisy y mi t\'io Alberto. A mis primos Ale, Franci, Yanila, Mabel y H\'ector. Gracias por el cari\~no, por celebrar mis logros como suyos, por impulsarme siempre para seguir adelante.
	
	A los que ya no est\'an, pero s\'e que desde el cielo velan por nosotros. A mis abuelas Chela y Cata y mis t\'ios Daisy y Hugo.
	
	A Betsy, Tanu, Claudia, Tedy y Ramses, mis hermanos, que a pesar de no habernos criado juntos han sentido suyas mis m\'as duras experiencias.
	
	A Ami, Sandri y Ariel por compartir conmigo alegr\'ias, l\'agrimas y locuras. A Yosy, Susana y Ana Laura por ser junto a Ami las hermanas que la Lenin me brind\'o. A Vitico por ser mi amigo m\'as viejo e incondicional. A todos por darme fuerzas cuando no ten\'ia, por nunca dejarme caer.
	
	A Nadia, Luis y Jose por las horas de oficina que se hicieron tan amenas.
	
	Al resto de esos amigos que la universidad me ha dado, lo m\'as bonito que me llevo de esta etapa de vida. A Carmen, Karla, Amanda, Aldo, Mederos, Osmany, Labu, Guaty, Kike, Pirlo, Henry, Javier, Laura, Sheyla, Thalia, Carlitos, Daniel, los Rodros y el team Santa F\'e.
	
	A mi tutor Camilo por no dudar dos veces cuando ped\'i su ayuda; por no preguntar razones siquiera antes de lanzarse a mi gu\'ia.
	
	A mis tutores Daniel, Katrib por el tiempo dedicado. Al profesor Luis Ramiro y al resto del departamento por su disposici\'on de ayudar. 
	
	A la profe Carmen por la confianza despositada en m\'i. Por mover cielo y tierra para hacer esta tesis una realidad.
	
	Al profesor Luciano por su tiempo y comprensi\'on. A la profe Lucina por tener siempre una palabra amorosa a la mano.
	
	Al resto de los docentes que hicieron estos a\~nos de carrera los m\'as maravillosos. Gracias por ense\~narnos m\'as que las materias b\'asicas, por la entrega en todo lo que hacen, por empujarnos a seguir adelante y no rendirnos jam\'as. A Juan Pablo, Wilfredo, Celia, Claudia, Camila, Fernando, Somoza, Piad y otros tantos que nos ense\~naron a so\~nar.
	
	Somos un pedacito de todos los que nos rodea. Es s\'olo con la compa\~nia de gente tan maravilosa que la vida toma color.
\end{acknowledgements}
\begin{opinion}
    El trabajo Módulo fabric-ca-client del SDK de Fabric para gestionar identidades criptogr\'aficas con C\# desarrollado por la estudiante Gabriela B. Martínez Giraldo, cumple con los requisitos para la culminación de la carrera de Ciencia de la Computación de la Universidad de La Habana.
   
    Este módulo es un componente de un SDK implementado en el lenguaje de programación C\# para interactuar con la tecnología de blockchain Hyperledger Fabric (HF). Siendo Hyperledger Fabric una de las plataformas usadas para el desarrollo de nuestras aplicaciones blockchain y siendo una debilidad la ausencia de bibliotecas, SDK y API en C\# hacen pertinente y útil el desarrollo de este módulo.
    
    Es válido aclarar que la diplomante ha mostrado excelentes habilidades técnicas durante el desarrollo del trabajo, demostrando interés, dominio del tema y cumpliendo con todos los requisitos definidos por el cliente. Para ello, comenzó con la asimilación y estudio de las tecnologías indicadas por los tutores, mostrando además buenas capacidades de asimilación e independencia
    
    Por tanto, felicitamos a la estudiante por la labor desarrollada y consideramos que la tesis reúne los estándares metodológicos exigidos por la Facultad de Matemática y Computación de la Universidad de la Habana, para ser presentada y sometida a evaluación en su ejercicio de defensa.
    
    Exhortamos a la estudiante a trabajar para convertir este trabajo en una publicación.
    
    La Habana, enero del 2022\\
    
    Camilo Denis González $\_\_\_\_\_\_\_\_\_\_\_\_\_\_\_$\\
    
    Camilo Denis González $\_\_\_\_\_\_\_\_\_\_\_\_\_\_\_$\\
    
    Miguel Katrib Mora $\_\_\_\_\_\_\_\_\_\_\_\_\_\_\_\_\_$\\
\end{opinion}
\begin{resumen}
	Resumen en español
	
	Debe ser muy breve debe tener como promedio
	250 palabras.
	No se trata de una presentación o relación de sus
	capítulos, sino de una exposición de los aspectos
	científicos esenciales contenidos en la tesis.
	Debe recordarse que el objetivo es informar al
	lector, en breves líneas, sobre el objeto y los
	objetivos del trabajo, sus resultados más relevantes
	y las contribuciones que hace a la ciencia o a la
	tecnología en el marco de su especialidad.
	Este mismo resumen debe colocarse en inglés.
\end{resumen}

\begin{abstract}
	Resumen en inglés
\end{abstract}
\tableofcontents
\listoffigures
% \listoftables
% \listofalgorithms
\lstlistoflistings

\mainmatter

\chapter*{Introducción}\label{chapter:introduction}
\addcontentsline{toc}{chapter}{Introducción}
% BibLaTeX + Biber

En los \'ultimos a\~nos se ha producido un incremento acelerado del uso y desarrollo de las \emph{blockchains}. Con m\'as de 900 publicaciones relacionadas hasta inicios del 2019 tan s\'olo en la \emph{Web of Science Core Collection WOC} [\cite{xu2019systematic} , \cite{yli2016current}], resulta sin dudas una tecnolog\'ia que ha irrumpido en el mundo como lo conocemos con intenciones de revolucionarlo. %sin intenciones de quedarse atr\'as.%co intenciones de revolucionarlo
%en la \emph{Web of Science Core Collection WOC}

%Se enfoca como Si bien... a la gente le atrajeron las cryptos... hay un monton de cosas bien cool en este ambito como el hecho de poder desarrollar contratos inteligentes, las apps encima por ello y el monton de aplicaciones q existen.. por lo cual se hace imprescindible en algunos casos como el de HF controlar quien accede a los datos sensibles q se guardan ahi
Si bien la primera propuesta de blockchain conocida data de 1982 [\cite{chaum1979computer}], no es hasta el 2009, con la apararici\'on del \emph{Bitcoin} [\cite{nakamoto2008bitcoin}], que realmente se le empieza a dar importancia. La idea de una moneda digital, independiente de instituciones poco confiables, con la cual fuera posible realizar transacciones an\'onimas pero a la vez cristalinas para todos los usuarios, ciertamente represent\'o, y representa a\'un, una gran competencia ante las industrias bancarias tradicionales. Sin embargo, las potencialidades que esta tecnolog\'ia tiene para ofrecer no se quedan a\'ca. %una alternativa mas atractiva que la de los ... os sistemas tradicionales de pago... las grandes transnacionales bancarias, Ahi se puede poner tambien un parrafito para cerrar diciendo que por ello no fue dificil cautivar al publico, aunque los mas escepticos aun duden.... dispuesta a permitir... que dicha tecnologia (con que cuenta)... no se limitan a estas o este sector

%blockchain data structures harden network security by reducing single-point-of-failure risk, making a database breach difficult.

%not feeling rigth about this sentence

%mencionar otras cryptos Cardano...

Desde la creaci\'on de Ethereum [\cite{ethereum2014ethereum}] como plataforma no s\'olo para criptomonedas, sino como ambiente para la programaci\'on y el despliegue de contratos inteligentes, comenzaron a desarrollarse un sinn\'umero de aplicaciones sobre este tipo de redes.

As\'i, las blockchains como registros digitales distribuidos, p\'ublicos e inmutables han mostrado su val\'ia en m\'ultiples esferas de la vida cotidiana. Se han aplicado a la medicina, la agricultura, el arte, la educaci\'on, la energ\'ia, para la verificaci\'on y el trazado de activos y el manejo de t\'itulos y escrituras [\cite{engelhardt2017hitching} , \cite{abou2019blockchain}].

%Pero, qu\'e sucede con el entorno empresarial?
%Blockchain helps in the verification and traceability of multistep transactions needing verification and traceability. It can provide secure transactions, reduce compliance costs, and speed up data transfer processing. Blockchain technology can help contract management and audit the origin of a product. It also can be used in voting platforms and managing titles and deeds.

%Internet of Things, energy, finance, healthcare, and government

N\'otese que la mayor\'ia de las blockchains conocidas son p\'ublicas o \emph{permissionless}, lo que significa que cualquiera puede interactuar en ellas. Sin embargo, en entornos del tipo empresarial se requiere en muchas ocasiones controlar qui\'enes tienen acceso a la red. Por ejemplo, regulaciones del tipo \emph{Know-Your-Customer KYC} y \emph{Anti-Money Laundering AML} en el ambiente finaciero exigen conocer qui\'en realiza cada transacci\'on. 

A raiz de estos y otros requerimientos funcionales, se han modificado numerosas redes existentes de anta\~no y se han creado otras espec\'ificas al sector empresarial. En este \'utltimo grupo se enmarca \emph{Hyperledger Fabric (HLF)} [\cite{androulaki2018hyperledger}], como una tecnolog\'ia de redes distribu\'idas (del ingl\'es \emph{Distributed Ledger Technology DLT}) dise\~nada para encajar en este panorama.%****Aqui es valido hacer la aclaracion de que DLT no es precisamente blockchain, sino q es mas amplio
%destacar sobresalir dominar

La naturaleza permisionada de HLF permite que los participantes de la red se conozcan entre s\'i, lo cual le aporta un gran nivel de seguridad. 
%A partir de esta informaci\'on los usuarios son capaces de elegir acorde a sus caracter\'isticas y necesidades reales un mecanismo de consenso  a partir de las posibilidades que brinda 
%puedan ser competidores es posibl elegir un mecanismo de consenso acorde a sus necesidades y caracter\'isticas reales.
Posee adem\'as una arquitectua modular y altamente configurable; y constituye una de las mejores plataformas disponibles en t\'erminos de procesamiento de transacciones y latencia de confirmaci\'on de las mismas. %plugging consensus protocol, no necesita de una criptomoneda nativa para incentivar el el minado, lo q a su vez reduce el riesgo de ataques

Una de sus peculiaridades m\'as relevantes, radica en el hecho de ser la primera DLT en brindar soporte para programar contratos inteligentes en lenguajes de prop\'osito general, en lugar de limitar el desarrollo a lenguajes espec\'ificos al dominio (DSL del ingl\'es Domain Specific Language) como lo es Solidity para Ethereum. Su flujo de ejecuci\'on del tipo \emph{execute-order-validate}, en contraste con el tradicional \emph{order-execute}, elimina el no determinismo, puesto que posibilita el filtrado de resultados incorrectos antes de pasar a la fase de ordenamiento. 

%De esta forma, en la actualidad %mas utilizado, sinonimo que ya esta palabra se repite arriba
%que hacen a HF \'unico 

Actualmente, Hyperledger [\cite{hyperledgerorg}] ofrece la capacidad de implementar \emph{smart contracts} en Go, Java y Node.js para interactuar con Hyperledger Fabric. Esto supone una facilidad tremenda para la empresa o individuo que desee trabajar con la DLT. Primero le exime de la necesidad de capacitaci\'on extra, al no requerir el estudio de nuevas herramientas. Segundo, le permite elegir, entre una gama de lenguajes, aquel que a su juicio provea las funcionalidades adecuadas al problema por resolver.%que le acate, que le corresponde %ofrece la posibilidad
% esta bien poner que es Hyperledger el que ofrece eso? Dar una clarification de quien es la empresa ( visto como an open source collaborative effort created to advance the implementation of blockchain technologies).

% grupo Blockchain del Instituto de Criptografía de la Facultad de Matemática y Computación de la Universidad de La Habana

El Instituto de Criptograf\'ia que radica en la Universidad de La Habana trabaja desde hace alg\'un tiempo en la investigaci\'on y desarrollo de proyectos vinculados a estas l\'ineas. En ese sentido se han efectuado ya los primeros acercamientos para extender la lista de lenguajes mencionados a otros de amplia utilizaci\'on como son Python [\cite{chaincode22python}] y C\# [\cite{chaincode22csharp}].
%radica en la UH? Aclarar esto
%notese que python ya tiene alguito pero de C\Sharp todavia nada, por tanto resulta novedoso

Este trabajo en particular pretende proseguir con el desarrollo de mecanismos que permitan la programaci\'on de contratos inteligentes en C\# para Hyperlegder Fabric.

Para dar continuidad a dicha investigaci\'on, resulta de inter\'es interactuar con el servidor de la autoridad certificadora de Fabric (Fabric-CA), a trav\'es de su REST-API, y as\'i administrar el ciclo de vida de los certificados de usuario. Dichos certificados son la base de la identificaci\'on de individuos mencionada con anterioridad. 

Las plataformas desarrolladas para el intercambio entre cada lenguaje particular y la blockchain de HLF, los Fabric-SDK, cuentan con un componente \emph{fabric-ca-client} que facilita a las aplicaciones registrar nodos y usuarios y establecer identidades confiables en la red. Luego, teniendo en cuenta la hipótesis de que “es viable la concepci\'on de un mecanismo similar a los existentes para extender las capacidades de interacci\'on con la red de Hyperledger Fabric al lenguaje de programaci\'on C\#”, es posible entonces exponer como objetivos
generales de esta tesis: “el dise\~nar e implementar un m\'odulo fabric-ca-client para interactuar con el Fabric-CA empleando C Sharp”.
Con el fin de dar cumplimiento a estos objetivos generales se pueden listar los siguientes objetivos específicos:

%También brinda soporte para envíos de transacciones seudónimas.
 
%como cada vez se hace mas necesario la creacion de aplicaciones que permitan interactuar con el mundo de la blochain y el de afuera. 

\begin{enumerate}
	\item Realizar un estudio de las bibliotecas fabric-sdk ya existentes para Python, Golang, Node.js y Java. Particularmente analizar el m\'odulo fabric-ca-client encargado de cumplir el prop\'osito de este proyecto. 
	
	\item Evaluar qu\'e estructuras y patrones de los ya existentes se pueden llevar a C\#, cu\'ales se pueden reutilizar y qu\'e otros no han sido consideradas a\'un.
	
	\item Concebir, dise\~nar, implementar el m\'odulo fabric-ca-client del fabric-sdk-csharp, para generar identidades criptogr\'aficas con CSharp. Como centro de la creaci\'on brindar soporte para las peticiones de registro, inscripci\'on, renovaci\'on y revocaci\'on de certificados.
	
	%aqui se pueden separar los objetivos en fab-ca-client y fab-ca-server mas c509Cert, pero puede ser excesivo
	
	\item Desarrollar un conjunto de pruebas sobre el m\'odulo implementado, de modo que sea posible comparar objetivamente los resultados con los que reportan otros lenguajes y valorar el cumplimiento de los requerimientos b\'asicos.
\end{enumerate}

El documento en cuesti\'on se estructura de la siguiente forma: El cap\'itulo \ref{chapter:state-of-the-art} realiza un bosquejo de las tecnolog\'ias relevantes para la comprensi\'on del presente estudio. El \ref{chapter:proposal} presenta la propuesta concreta a desarrollar, describe los m\'odulos y sus funcionalidades b\'asicas. El cap\'itulo \ref{chapter:implementation} comenta los resultados obtenidos y las pruebas realizadas. Finalmente, se exponen los \'ultimos an\'alisis y potenciales directivas para encaminar investigaciones futuras.

%Las aplicaciones interactuan con el ledger a traves de un intermediario que seria un sdk, el que a su vez se comunica con el smart contract que es el que realiza las operaciones 'get, put, delete' sobre el world state y registra las transacciones en la blokchain.

%(https://www.bbva.com/es/diferencia-dlt-blockchain/)

%Es habitual pensar que todas las ‘blockchain’ son DLT, es decir, tecnologías de registro distribuido (‘Distributed Ledger Technology’, en inglés), sin embargo,una ‘blockchain’, una cadena de bloques, es un tipo de DLT.Desde un punto de vista más técnico, una DLT es simplemente una base de datos que gestionan varios participantes y no está centralizada. ‘Blockchain’ es una DLT con una serie de características particulares. También es una base de datos —o registro— compartida, pero en este caso mediante unos bloques que, como indica su propio nombre, forman una cadena. Los bloques se cierran con una especie de firma criptográfica llamada ‘hash’; el siguiente bloque se abre con ese ‘hash’, a modo de sello lacrado. De esta forma, se certifica que la información, encriptada, no se ha manipulado ni se puede manipular.

\chapter{Antecedentes}\label{chapter:state-of-the-art}

En el presente cap\'itulo se ofrecen una serie de conceptos relevantes al tema en cuesti\'on. Su prop\'osito es el de facilitar la comprensi\'on de las bases en las que se enmarca esta investigaci\'on. %este proyecto

La mayor\'ia de las definiciones que se proveen tienen como fuente \cite{hyperledgerPaper}, \cite{Hyperledgerdoc} y \cite{HyperledgerdocCA}. Para el resto se esclarecen las referencias apropiadas.%conceptos, se presentan, definiciones provienen de % adecuadas % debidamente referenciados %referencias correspondientes

\section{Blockchain}

%https://hyperledger-fabric.readthedocs.io/en/release-2.5/whatis.html
Una blockchain o cadena de bloques no es m\'as que un libro mayor de transacciones inmutable y distribuido entre una red de nodos. Cada nodo mantiene una copia del \emph{ledger}, la cual se actualiza por medio de transacciones ejecutadas y validadas a partir de un protocolo de consenso. La informaci\'on (transacciones) se agrupa en bloques que incluyen un hash del bloque precedente, de forma que resulta casi imposible alterar los datos de una transacci\'on sin falsificar tambi\'en la de los bloques siguientes.

AQUI SE PUEDEN PONER VENTAJAS DE LAS BLOCKCHAINS O LAS CARACTERISTICAS MAS GENERALES< O BIEN INFO DEL TIMELINE 

PUEDE SER O BIEN EXPLICAR QUE ES EL CONSENSO< UN POQUITO MAS DE PQ ES INALTERABLE< ETC

%https://www.geeksforgeeks.org/consensus-algorithms-in-blockchain/

There is no central authority present to validate and verify the transactions, yet every transaction in the Blockchain is considered to be completely secured and verified. This is possible only because of the presence of the consensus protocol which is a core part of any Blockchain network. A consensus algorithm is a procedure through which all the peers of the Blockchain network reach a common agreement about the present state of the distributed ledger. In this way, consensus algorithms achieve reliability in the Blockchain network and establish trust between unknown peers in a distributed computing environment. Essentially, the consensus protocol makes sure that every new block that is added to the Blockchain is the one and only version of the truth that is agreed upon by all the nodes in the Blockchain. 

immutability, privacy, security, and transparency
%----------< HF Aitentication Authorization
%Combining ID’s, Attributes, and Policies in Hyperledger Fabric Through the use of innovative concepts such as channels, policies, identities, and Membership Service Providers, Hyperledger Fabric can determine the identity of participants, perform access control based on these identities, and ensure the privacy of transactions and smart contracts.
%----------> HF Aitentication Authorization

%\section{Blockchain Timeline}

\section{Understanding HLF}

Hyperledger Fabric es un proyecto de c\'odigo libre del consorcio Hyperledger, creado su vez por la \emph{Linux Foundation} [\cite{linuxFoundation}] con el objetivo avanzar en el desarrollo de tecnolog\'ias blockchain. Consiste en una DLT dise\~nada espec\'ificamente para entornos empresariales, atendiendo a las necesidades de identificar a los participantes en la red, establecer una serie de permisos para la interacci\'on, de contar con privacidad y confidencialidad de trasacciones, baja latencia en la confirmaci\'on de estas y capacidad para enviar una cantidad elevada de datos por unidad de tiempo (\emph{throughput}).

AHORA LISTAR CARACTERISTICAS< PQ LO Q SE MENCIONARON FUERON NECESIDADES
aqui mencionar tambien la exsitencia de canales para manejar info probada
Where Hyperledger Fabric breaks from some other blockchain systems is that it is private and permissioned. Rather than an open permissionless system that allows unknown identities to participate in the network (requiring protocols like “proof of work” to validate transactions and secure the network), the members of a Hyperledger Fabric network enroll through a trusted Membership Service Provider (MSP).

HF es una plataforma permisionada, ideal para contextos empresariales donde se necesita identificar a los usuarios en la red y permititr que solo un numero de usuarios
actualizados accedan a esta. 

Posee una arquitectura altamente modular y configurable, capaz de ser adaptada a los diversos requerimientos que puedan surgir en la industria. Facilita customizar por ejemplo el protocolo de consenso a utilizar para determinar el orden de las transacciones. Con este fin provee implementaciones de mecanismos crash fault-tolerant (CFT) como Raft (el actual y recomendado), Kafka y Solo (para versiones de prueba), lo cual permite a los usuarios elegir el m\'as adecuado al nivel de confianza existente entre ellos.
%Aqui pueden referenciarse varios articulos sobre los mecanismo de ordenaicon, solo que considero no son tan relevantes al tema tratado

De igual forma cuenta con protocolos enchufables de gesti\'on de identidad como LDAP u OpenID Connect, con protocolos de gesti\'on de claves o bibliotecas criptogr\'aficas, entre otros.

N\'otese adem\'as que, al utilizar mecanismos de consenso deterministas no requiere del uso de criptomonedas para incentivar el minado, lo cual resulta en una seguridad incrementada al no atraer atacantes que quieran socavar la estabilidad de la red. Adem\'as implica un costo operacional menos elevado de la plataforma .

\subsection{Componentes principales}

\subsubsection{Ledger}
El ledger o libro mayor de cuentas de HF est\'a conformado por dos elementos: Un \emph{world state} o estado mundial y un \emph{transaction log} o lista de trasacciones. El primero resulta ser como la base de datos del ledger, describe el estado de este en un instante de tiempo dado. Por su parte, el segundo, mantiene un registro del historial de trasacciones que han resultado en el estado actual del world state.

Cada participante mantiene una copia del ledger por cada red de HF a la que pertenece.

\subsubsection{Assets}
Ha de resaltarse que, mientras en blockchains como Bitcoin se mantiene un registro de las transferencias de criptomonedas, en Fabric la definici\'on de qu\'e es transferido resulta un poco m\'as flexible. En el \'ultimo caso se definen activos o \emph{assets} a transferir, lo que da cabida a un n\'umero de usos m\'as extendido.
%aqui se puede argumentar

Los activos pueden variar desde elementos tangibles como bienes ra\'ices y hardware, hasta intangibles como contratos y propiedades intelectuales. Hyperledger Fabric brinda la capacidad de modificar activos mediante transacciones de c\'odigo de cadena.

Los activos representan precisamente la colecci\'on de pares clave-valor, cuyos cambios de estado se registran como transacciones en el libro mayor.

\subsubsection{Chaincode}

Los \emph{smart contracts} son llamados \emph{chiancode} o c\'odigo de cadena en Fabric. Estos constituyen la l\'ogica de negocio de una aplicaci\'on blockchain. 

Los chaincodes definen activo(s) y transacciones para modificarlos. Pueden ser desplegados din\'amicamente y correr de forma concurrente en la red.

Los elementos participantes de cada red permisionada en Fabric son capaces de utilizarlos para interactuar con el ledger y transformarlo.

\subsubsection{Nodos}

Los nodos constituyen las entidades de comunicaci\'on de una blockchain. En HF existen 3 tipos:

\begin{enumerate}
	\item Nodos Clientes: Los clientes que env\'ian invocaciones de trasacciones a los \emph{endorsers} y transmiten las proposiciones de trasacciones al servicio de ordenaci\'on.
	
	
	\item Nodos Pares: O \emph{peers}, qui\'enes ejecutan trasacciones y mantienen el estado y una copia del ledger. Pueden asumir el rol de endosadores o endorsers.
	
	\item Nodos Ordenadores: El conjunto de estos nodos conforman lo que se le llama servicio de ordenamiento, el cual se encarga de ordernar las trasacciones endosadas y empaquetarlas en bloques. De igual forma se ocupan de darle un orden a dichos bloques siguiendo un protocolo de consenso como se comentaba anteriormente. Su segundo objetivo consiste en distribuir dichos bloques entre los peers para su posterior validaci\'on.
	
\end{enumerate}
%Ordering-service-node or orderer: a node running the communication service that implements a delivery guarantee, such as atomic or total order broadcast.

Aqu\'i es importante destacar c\'omo el hecho de separar el endosamiento de ejecuci\'on de chaincode (en los peers) del servicio de ordenamiento, al eliminar el embotellamiento, le provee a HF ventajas de desempe\~no y escalabilidad sobre los sistemas que ejecutan ambas fuciones en los mismos nodos.

CAnales


%Modularity Hyperledger Fabric has been specifically architected to have a modular architecture. Whether it is pluggable consensus, pluggable identity management protocols such as LDAP or OpenID Connect, key management protocols or cryptographic libraries, the platform has been designed at its core to be configured to meet the diversity of enterprise use case requirements. At a high level, Fabric is comprised of the following modular components: 
%A pluggable ordering service establishes consensus on the order of transactions and then broadcasts blocks to peers.
%------A pluggable membership service provider is responsible for associating entities in the network with cryptographic identities.
%-----An optional peer-to-peer gossip service disseminates the blocks output by ordering service to other peers.
%Smart contracts (“chaincode”) run within a container environment (e.g. Docker) for isolation. They can be written in standard programming languages but do not have direct access to the ledger state.
%------The ledger can be configured to support a variety of DBMSs.
%------A pluggable endorsement and validation policy enforcement that can be independently configured per application.
 
\subsection{Flujo}

La mayor\'ia de las blockchains utilizan un paradigma nombrado \emph{order-execute} que requiere, para alcanzar el consenso, que los smart contracts a utilizar sean deterministas. Lo anterior limita su programaci\'on a lenguajes de dominio espec\'ifico que eliminen las operaciones no deterministas. Adem\'as, require que todas las transacciones se ejecuten secuencialmente por todos los nodos, lo que implica que deban ser tomadas medias de protecci\'on extra ante posibles ataques maliciosos que deseen desequilibrar la red.

En cambio, en Fabric se utiliza un arquitectura llamada \emph{order-execute-validate} que resuelve muchos de los problemas comentados.

Dicho modelo consta de tres pasos:
\begin{enumerate}
	\item Primeramente se ejecuta una trasacci\'on, se chequea su correctitud y se respalda (proceso de \emph{endorsement}).
	
	\item Luego se ordenan las trasacciones a partir del protocolo de consenso.
	
	\item Se validan teniendo en cuenta la pol\'itica de endosamiento espec\'ifica a la aplicaci\'on manejada y, por \'ultimo, se aplican al ledger.
\end{enumerate}

Las pol\'iticas de endosamiento a las que se hace referencia dictan qu\'e nodos par y cu\'antos de ellos deben garantizar la ejecuci\'on correcta de un contrato inteligente determinado. De esta forma se asegura que s\'olo un subcojunto de los peers deba ejecutarlo. Esto facilita que el sistema sea m\'as escalable y tenga un mejor desempe\~no con la capacidad a\~nadida de ejecuciones en paralelo.

Igualmente, ha de notarse c\'omo se elimimina el no determinismo, al poder filtrar los resultados inconsistentes antes de pasar a la fase de ordering.

Dicho cambio de arquitectura contituye una de las caracter\'isticas que realzan la importancia de HF. Es este nuevo modelo que implementa, el que hace posible la programaci\'on de chaincode en lenguajes est\'andar como Python, Java y Javascript. V\'ease su grandeza de facilitar al programador el crear un v\'inculo con esta nueva DLT, ahorr\'andole horas de estudio para aprender un nuevo lenguaje de programaci\'on y de debugueo para encontrar errores nunca tratados por \'el.

Precisamente en esta capacidad de Fabric es en la que se apoya este proyecto, puesto que es la que hace posible la programaci\'on de smart contracts en C\# tras implementar el engranaje apropiado.
 
%The order-execute architecture can be found in virtually all existing blockchain systems, ranging from public/permissionless platforms such as Ethereum (with PoW-based consensus) to permissioned platforms such as Tendermint, Chain, and Quorum.

%This design departs radically from the order-execute paradigm in that Fabric executes transactions before reaching final agreement on their order.

%\subsubsection{Ejemplo}

%La arquitectura de HF cuenta con un SDk que es el que permite la comunicacion entre las aplciaciones y la red blockchain. Cuando por ejemplo un cliente A desea comprar comida de uno B: (descrito en la diapo 7 de la conf Intro del tema 3)

%-El cliente A (reconocido por el Fabric CA, el encargado de la autenticacion) realiza una peticion para comprar comida (mediante un canal que les permite crear un ledger separado y tener un mecanismo para comunicaciones privadas y datos privados). 
%-El sdk la recibe y crea una propuesta de transacción a los nodos peers (que tienen el chaincode instalado). 
%-Los pares-endosadores comprueban la propuesta de transacción y ejecutan la transacción, pero NO ACTUALIZAN el ledger, sino que envían de vuelta una propuesta de respuesta (firma + conjunto delta).
%-El SDK del nodo-cliente valida las firmas de los nodos-pares que firmaron y compara las respuestas de las propuestas
%-El nodo-cliente difunde la “propuesta de transacción” y la respuesta al “servicio de ordenamiento”.
%-El “servicio de ordenamiento” recibe transacciones de todos los canales de la red.
%-Ordena transacciones por canal, y crea bloques por canal. os bloques se envían a todos los nodos-pares del canal.
%- Cada nodo-par valida que no existan cambios.
%- Cada nodo-par etiqueta cada transacción del bloque como válidas o no válidas.
%- Cada nodo-par agrega un bloque a la cadena del canal
%- Para transacciones válidas, los conjuntos delta se agregan a la base de datos de estado.

\section{Fabric SDKs} 

Las aplicaciones que deseen interactuar con el ledger lo deben hacer por medio de un intermediario, un SDK o paquete de desarrollo de \emph{software}. Los espec\'ificos a HF est\'an dise\~nados para comunicarse con los smart contracts, encargados a su vez de realizar las operaciones de \emph{get, put, delete} sobre el world state y registrar las trasacciones en la blockchain.

Hyperledger Fabric ofrece un n\'umero de SDKs para el desarrollo de aplicaciones en una amplia variedad de lenguajes de programaci\'on. Actualmente cuenta con soporte para Node.js [\cite{SDKNode}] y Java [\cite{SDKJava}], los primeros en aparecer. Igualmente, a pesar de a\'un no haber sido publicados de forma oficial, expone para su descarga y prueba implementaciones para Go [\cite{SDKGO}] y Python [\cite{SDKPython}].

Vale aclarar que estos paquetes son considerados SDKs de bajo nivel. Contienen una considerable cantidad de c\'odigo y, por tanto, una complejidad bastante elevada. Queda en manos programador con la intenci\'on de conectar una aplicaci\'on con Fabric, comprender el funcionamiento de las solicitudes de transacciones y el endosamiento para implementarlos a partir de estas plataformas, dado que en realidad lo que contienen no es m\'as que una adaptaci\'on de las llamadas gRPC [\cite{grpc}] para cada lenguaje espec\'ifico.

Para resolver estos problemas se han creado APIs de alto nivel que le permiten a los desarrolladores abstraerse un poco m\'as del funcionamiento de HF en sus or\'igenes. As\'i, con proyectos como \cite{FabricGatewayJava} y \cite{FabricGatewayGeneral} se facilita la realizaci\'on de consultas, ejecuci\'on de transacciones y respuesta a eventos que permitan conocer y modificar el estado del ledger.

El surgimiento de estos modelos m\'as consistentes, supresores de mucho c\'odigo repetitivo propenso a errores, ha sido bien recibido por la comunidad. No obstante, su funcionamiento cuenta en las bases con los SDKs mencionados al inicio, siendo de hecho construidos encima de estos. Adem\'as carecen de un mecanismo para gestionar tareas de administraci\'on. Por ello, a\'un se hace necesario intercambiar directamente con los SDKs de bajo nivel para la creaci\'on de identidades, su registro entre otras operaciones de relevancia.

Dado que el objetivo del presente trabajo gira en torno a estas funcionales, son entonces los SDKs originales de Java, Node, Go y Python los que fungen como gu\'ia para el desarrollo.

\chapter{Propuesta}\label{chapter:proposal}

\chapter{Detalles de Implementación y Experimentos}\label{chapter:implementation}


\backmatter

\begin{conclusions}
    Conclusiones
\end{conclusions}

\begin{recomendations}
    Recomendaciones
    Deben ser elaboradas atendiendo a las siguientes
    cuestiones:
    • En qué aspectos se propone trabajar más para
    completar o ampliar el trabajo realizado.
    • La posibilidad, a partir del trabajo realizado, de
    resolver problemas similares en otras esferas.
    • Consideraciones sobre las condiciones necesarias
    para introducir los resultados a la práctica
    pedagógica.
    
    Anexos
    
    • Se presentan al final del informe.
    • Cada anexo debe estar titulado y numerado 
    consecutivamente en correspondencia con su 
    referencia en el informe. 
    • En los anexos se incluye el material auxiliar que no 
    aparece en el texto del informe, se incluyen por lo 
    general: tablas, gráficos, ilustraciones, algoritmos, 
    fragmentos de código, etc.
\end{recomendations}

%\printbibliography[heading=bibintoc]

\printnomenclature

\end{document}